\chapter{Java 2 Micro Edition}
\label{sec:Java Micro Edition}
%
\section{Wstęp}
Na pocz¹tku lat dziewiêædziesi¹tych w firmie Sun Microsystems
powsta³ nowy jêzyk programowania nazwany Oak. Stanowi³ on czeœæ
projektu badawczego zajmuj¹cego siê urz¹dzeniami elektroniki
u¿ytkowej intensywnie wykorzystuj¹cej oprogramowanie. Pierwszym prototypem wykorzystuj¹cym kod Oak, by³ kontroler Star7. by³o to
niewielkie urz¹dzenie przenoœne wyposa¿one w ekran dotykowy LCD,
wbudowany interfejs sieci bezprzewodowej oraz ³¹cze na podczerwieñ.
Urz¹dzenie to spe³nia³o funkcje elektronicznego notesu, organizera,
mog³o te¿ byæ wykorzystywane jako pilot do telewizora, magnetowidu
itp. oprogramowanie tego typu urz¹dzeñ musi byæ bardzo niezawodne i
nie powinno wymagaæ do pracy zbyt wiele pamiêci ini wydajnego
procesora, czyli elementów które ze wzglêdu na rozmiary urz¹dzenia
zwielokrotniaj¹ jego cenê. Oak powsta³ na bazie doœwiadczeñ
programistów C++ i mia³ podobne mo¿liwoœci, jednak by³ o wiele
bardziej wra¿liwy na b³êdy programistów. Z za³o¿enia jêzyk Oak mia³
wyeliminowaæ ca³kowicie b³êdy w programach poprzez ograniczenie
mo¿liwoœci ich pope³niania. Poniewa¿ Oak nie toleruje b³êdów, s¹ one
wykrywane w procesie kompilacji, usuniêto równie¿ niektóre funkcje
jak np wskaŸniki i zarz¹dzanie pamiêci¹. Niestety w latach kiedy
powstawa³ Oak rynek urz¹dzeñ mobilnych praktycznie nie istnia³,
równie¿ nie rozwin¹³ siê tak szybko jak przewidywali programiœci,
dlatego te¿ nigdy nie sprzedano urz¹dzenia korzystaj¹cego z Oak.


 Z pomoc¹ twórcom Oak przyszed³ Internet. gwa³townie rosn¹ca
popularnoœæ internetu spowodowa³a zapotrzebowanie na oprogramowanie
do przegl¹dania zasobów globalnej sieci. W odpowiedzi firma Sun
zmieni³a nazwê Oak na Java i zaimplementowa³a "nowy" jêzyk w
przegl¹darce o nazwie HotJava. Dodatkowo licencjê na jêzyk Java
wykupi³a firma Netscape której przegl¹darka by³a w owym czasie
niekwestionowanym liderem na rynku.


Z biegiem czasu gwa³townie ros³y zasoby sprzêtowe. Aby sprostaæ
wymaganiom programistów Windows tworz¹cych bardzo zaawansowane
aplikacje, platforma Java stale siê rozszerza³a, pojawia³y siê nowe
funkcje, programowanie rozproszone, lepsze mechanizmy
bezpieczeñstwa.


Podczas gdy coraz wiêkszym zainteresowaniem cieszy³a siê przenoœnoœæ
czyli inaczej mobilnoœæ urz¹dzeñ, Java udostêpni³a now¹ rozszerzon¹
platformê o nazwie Java 2. niezbêdne sta³o siê podzielenie platformy
Java 2 na kilka czêœci. Postawowe funkcje uwa¿anie za niezbêdne
minimum umieszczone zosta³y w jednym pakiecie i zyska³y nazwê Java
2 Standard Edition (J2SE). Do J2SE dodano kilka funkcji na potrzeby
okreœlonych zastosowañ korporacyjnych jak np. bezpieczna komunikacja
sieciowa, handel elektroniczny i tak powsta³a Java 2 Enterprise
Edition (J2EE).

Ze wzglêdu na urz¹dzenia o ograniczonych mo¿liwoœciach, konieczne
sta³o siê ograniczenie tej rozbudowanej platformy Java.
Paradoksalnie Oak, prototyp jêzyka Java powsta³ w³aœnie na potrzeby
urz¹dzeñ mobilnych. Podczas gdy zapotrzebowanie na tego typu
platformê faktycznie zaistnia³o, platforma Java by³a ju¿ na tyle
rozbudowana ¿e nie mog³a zostaæ zastosowana w urz¹dzeniach
przenoœnych w takiej formie. Zatem programiœci Java korzystaj¹c z
doœwiadczeñ z jêzykiem Oak stworzyli kilka platform o ograniczonych
mo¿liwoœciach. Powrócono do platformy JDK 1.1 poprzednika Java 2 na
których oparto min. Java 2 Micro Edition (J2ME)


\section{Klasyfikacja CDC, CLDC i KVM}

Ze wzglêdu na ró¿nice w zasobach urz¹dzeñ jak np. wydajnoœæ
procesora, iloœæ pamiêci, sposób nawi¹zywania po³¹czeñ J2ME
podzielona jest na dwie podstawowe konfiguracje CDC i CLDC.

\subsection{CLDC}
\emph{Connected Limited Device Configuration}

Konfiguracja ta jest przeznaczona dla najprostszych urz¹dzeñ
elektroniki u¿ytkowej. Typowa platforma CLDC to telefon komórkowy
lub notes wyposa¿ony w oko³o 512kB pamiêci. Z tego powodu CLDC jest
œciœle zwi¹zany ze specyfikacj¹ bezprzewodowej Javy, która umo¿liwia
instalowanie niewielkich aplikacji zwanych MIDletami. Wiêkszoœæ firm
produkuj¹cych telefony komórkowe podpisa³o porozumienie z firm¹ Java
Microsystems które umo¿liwia im stosowanie tej technologii, przez co
platforma Java mo¿e byæ z powodzeniem wykorzystywana przez prawie
wszystkie dostêpne na rynku modele telefonów komórkowych i jej
popularnoœæ stale roœnie.
\subsection{CDC}
\emph{Connected Device Configuration}

 Konfiguracja ta jest adresowana do urz¹dzeñ znajduj¹cych siê
 pomiêdzy urz¹dzeniami CLDC a normalnymi systemami, na których mo¿na
 uruchomiæ J2SE (np. komputer klasy PC). Zwykle urz¹dzenia takie
 maja wiêcej pamiêci (minimum 2MB) i nieco wydajniejszy procesor,
 dziêki czemu mog¹ obs³ugiwaæ bardziej kompletne œrodowisko Java.
 Konfiguracja CDC jest stosowana w bardziej zaawansowanych notesach
 elektronicznych, smartfonach, telefonach sieciowych, urz¹dzeniach
 sterowania oraz zestawach audio video.

\section{Profil informacji o urz¹dzeniu przenoœnym, MIDP}

bla bla bla

\section{Struktura MIDletu}

\section{Media API}


    \subsection{Rejestrator}
    \label{sec:Rejestrator}
    \subsection{Kodowanie}
    \label{sec:Kodowanie}
    \subsection{Odtwarzanie}
    \label{sec:Odtwarzanie}

\section{Bluetooth API}
\label{sec:Bluetooth API}

    \subsection{Discovery Agent}
    \subsection{Nawi¹zywanie po³¹czenia}
    \subsection{Streaming}
    \subsection{Przesy³anie danych}

\section{Uruchamianie}
\label{sec:Uruchamianie}

  \subsection{Kompilacja}
    \label{sec:Kompilacja}
    \subsection{Symulacja}
    \label{sec:Symulacja}

\section{Bezpieczeñstwo}


    \subsection{Autoryzacja}
    \label{sec:Autoryzacja}
    \subsection{Podpisywanie MIDletów}
    \label{sec:Podpisywanie MIDletow}

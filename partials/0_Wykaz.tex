\chapter*{Wykaz ważniejszych oznaczeñ}
{\small
\begin{tabular}{ccp{10cm}}
  $\bf{A}$ & -- & magnetyczny potencja³ wektorowy\\
  $AF$ & -- & charakterystyka grupowa dyskretnego uk³adu antenowego\\
  $A,~B$ & -- & funkcja\\
  $c$ & -- & prêdkoœæ œwiat³a w~pró¿ni\\
  $d$ & -- & wysokoœæ modu³u anteny, odleg³oœæ pomiêdzy Ÿród³ami modelu dyskretnego\\
  $D$ & -- & najwiêkszy wymiar liniowy anteny\\
  $\bf{D}$ & -- & wspó³czynnik dyfrakcji\\
  ${\bf E}$ & -- & zespolony wektor natê¿enia pola elektrycznego\\
  ${\bf E}_i$ & -- & sk³adowa indukcyjna pola elektrycznego\\
  ${\bf E}_p$ & -- & sk³adowa promieniowania pola elektrycznego\\
  $E$ & -- & wartoœæ skuteczna natê¿enia pola elektrycznego\\
  ${\bf E}^p$ & -- & sk³adowa elektryczna pola pierwotnego\\
  ${\bf E}^w$ & -- & sk³adowa elektryczna pola wtórnego\\
  $e$ & -- & podstawa logarytmu naturalnego \\
  $\bf{F}$ & -- & elektryczny potencja³ wektorowy\\
  $F$ & -- & ca³ka Fresnela\\
  $f$ & -- & czêstotliwoœæ, funkcja\\
  $[F],~[G]$ & -- & wektor kolumnowy\\
  F/B & -- & stosunek promieniowania przód/ty³\\
  $F_{\: \rm m}(\theta,\phi)$ & -- &
  charakterystyka promieniowania odosobnionego modu³u
  anteny dla pola elektrycznego\\
  $F_{\: \rm V}(\theta)$ & -- &
  charakterystyka promieniowania odosobnionego modu³u
  anteny w~p³a\-szczy\-Ÿnie pionowej\\
  $F_{\: \rm H}(\phi)$ & -- &
  charakterystyka promieniowania odosobnionego modu³u
  anteny w~p³a\-szczy\-Ÿnie poziomej\\
  $g$ & -- & funkcja\\
  $G(\phi, \theta)$ & -- & funkcja zysku energetycznego anteny\\
  $G$ & -- & zysk energetyczny\\
  $G_c(\phi)$ & -- & funkcja zysku energetycznego dla fali cylindrycznej\\
  $G_c$ & -- & zysk energetyczny Ÿród³a fali cylindrycznej\\
  $g_{\rm H}(\phi)$ & -- & charakterystyka promieniowania mocy w~p³a\-szczy\-Ÿnie poziomej\\
  $G_{\rm m}(\phi, \theta)$ & -- & funkcja zysku energetycznego odosobnionego modu³u anteny
      \end{tabular}
} \newpage  \thispagestyle{empty}
 {\small
\begin{tabular}{lcp{9cm}}
  $G_{\rm m}$ & -- & zysk energetyczny odosobnionego modu³u anteny\\
  $h$ & -- & wysokoϾ anteny\\
  $\bf{H}$ & -- & zespolony wektor natê¿enia pola magnetycznego\\
  $\bf{H}_i$ & -- & sk³adowa indukcyjna pola magnetycznego\\
  $\bf{H}_p$ & -- & sk³adowa promieniowania pola magnetycznego\\
  $H$ & -- & wartoœæ skuteczna natê¿enia pola magnetycznego\\
  ${\bf H}^p$ & -- & sk³adowa magnetyczna pola pierwotnego\\
  ${\bf H}^w$ & -- & sk³adowa magnetyczna pola wtórnego\\
  $[I]$ & -- & wektor pr¹du \\
  $j$ & -- & jednostka urojona \\
  $k$ & -- & liczba falowa \\
  ${\bf J}_s$ & -- & gêstoœæ elektrycznego pr¹du powierzchniowego\\
  $[L]$ & -- & macierz kwadratowa\\
  $L$ & -- & operator liniowy\\
  $l$ & -- & d³ugoœæ\\
  ${\bf M}_s$ & -- & gêstoœæ magnetycznego pr¹du powierzchniowego\\
  $P$ & -- & moc wypromieniowana przez antenê\\
  ${\bf r}$ & -- & wektor wodz¹cy \\
  $r_{\rm d}$ & -- & granica obszaru pola dalekiego odosobnionego
  modu³u anteny\\
  $R,~r$ & -- & odleg³oœæ\\
  $R$ & -- & wspó³czynnik odbicia \\
  $\bf{S}$ & -- & zespolony wektor gêstoœci strumienia mocy\\
  $S_c$ & -- & wartoœæ œrednia gêstoœci strumienia mocy fali cylindrycznej\\
  $\bf{S}_p$ & -- & wektor gêstoœci strumienia mocy skojarzony z~polem promieniowania\\
  $S$ & -- & wartoœæ œrednia gêstoœci strumienia mocy\\
  $s$ & -- & odleg³oœæ \\
  SAR & -- & tempo poch³aniania na jednostkê masy\\
  $T$ & -- & wspó³czynnik transmisji\\
  $u$ & -- & funkcja, prêdkoœæ rozchodzenia siê fali w~oœrodku\\
  $[V]$ & -- & wektor pobudzenia \\
  $w$ & -- & szerokoϾ anteny\\
  $w_n$ & -- & $n$-ta funkcja wagowa\\
  $x,~y,~z,$ & -- & wspó³rzêdne prostok¹tne punktu \\
  $[Z]$ & -- & macierz impedancyjna tworzona metod¹ momentów \\
  ${\bf 1}_{p}$ & -- & wektor jednostkowy okreœlaj¹cy polaryzacjê pola \\%wytwarzanego przez $i$-te Ÿród³o\\
  $\nabla$ & -- & operator nabla\\
  $\alpha$ & -- & t³umiennoœæ oœrodka\\
  $\alpha_n$ & -- & $n$-ty wspó³czynnik rozwiniêcia\\
  $\Delta x$ & -- & krok dyskretyzacji przestrzeni\\
  $\Delta t$ & -- & krok dyskretyzacji czasu\\
  $\epsilon$ & -- & przenikalnoϾ elektryczna \\
  $\varepsilon$ & -- & b³¹d wzglêdny
 \end{tabular}
}
\newpage
\thispagestyle{empty}%<---------------------------- pusta strona na odwrocie tutu³owej
%\pagestyle{fancy}
\cleardoublepage%<--------------------------------- dalej bêdzie od nieparzystej

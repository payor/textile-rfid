
\singlespacing
\chapter{Wstęp}

\section{Od kodu kreskowego do tagu radiowego}

Obserwujemy na przestrzeni ostatnich kulkudziesieciu lat znaczący rozwój technologii bezprzewodowej oraz mobilnych form płatności elektronicznej spowodował, żę obecnie niemal na każdym kroku spotykamy się z systemami automatycznej identyfikacji (ang. \emph{Auto ID Systems}). Systemy te umożliwiają szybką, wygodną i bezbłędną identyfikacje produktów, sprzętu, ludzi a nawet zwierząt. Przekłada się to zazwyczaj na poprawę poziomu życia oraz jakości świadczonych usług spedycyjnych. Podstawowym zadaniem systemów automatycznej identyfikacji jest sprawne, przeprowadzone w czasie rzeczywistym i bez udziału człowieka, zarządzanie łańcuchem dostaw oraz kontrola jednostek logistycznych. Ostatnio systemy te coraz częściej wykorzystuje się do autoryzacji transakcji bankowych, realizowanych za pośrednictwem sieci telefonii komórkowej, sieci przewodowej LAN i bezprzewodowej WLAN.


\noindent 
\newline Systemy automatycznej identyfikacji obejmują:
\begin{itemize}\setlength{\itemsep}{0pt}
    \item systemy kodów kreskoweych (ang. \emph{bar code}),
    \item systemy kart elektronicznych (ang. \emph{electronic cards}),ścieżki magnetyczne (ang. \emph{magnetic stripe}),
    \item systemy RFID (ang. \emph{Radio Frequency Identyfication}),
    \item systemy automatycznego rozpoznawania pisma ręcznego ICR (ang. \emph{Intelligent Charakter Recognition}), 
    \item systemy automatycznego rozpoznawanie znaków (druku) OCR (ang. \emph{Optical Character Recognition}),
    \item systemy biometryczne bazujące na technice:
	\begin{itemize}\setlength{\itemsep}{0pt}
		\item identyfikacji linii papilarnych,
		\item identyfikacji głosu,
		\item identyfikacji siatkówki oka,
		\item rozpoznawaniu twarzy
	\end{itemize}
\end{itemize}

Za protoplastę systemów automatycznej identyfikacji zwykło się uważać systemy kodów kreskowych, wprowadzone do powszechnego użytkuw latach 60-tych XX wieku.
Kod kreskowy nazywany często potocznie kodem paskowym (ang. \emph{bar code}) stanowi szereg ciemnych i jasnych pasków o zróżnicowanej szerokości, wykorzystywanych do zakodowania danych cyfrowych. Wykorzystując własności statyczne nośnika informacji, w tym przypadku białej i czarnej farby tworzącej pasek kodu kreskowego, jesteśmy w stanie szybko i w sposób jednoznaczny dokonać identyfikacyji skanowanego obiektu. Emitowane przez czytnik światło zostaje odbite od jasnych pasków, a pochłonięte przez paski czarne. Dekodowanie informacji cyfrowej zapisanej w kodzie kreskowym sprowadza sie do porównania, mierzonego w luksach natężenia światła emitowanego przez czytnik z natężeniem światłao dbitym przez kod kreskowy. Zastosowanie odpowiedniego fotodetektora pozwala tą różnicę w poziomie natężenia światła przekształcić w ciąg impulsów elektrycznych, które są następnie przesyłane do komputera. W komputerze impulsy te są przetwarzane na ciąg odpowiednich znaków alfanumerycznych zapisywanych w bazie danych. 
Kody kreskowe są powszechnym i tanim sposobem etykietowania produktów. Zaletą tego systemu jest powszechna unifikacja sposobu kodowania. Wadami  natomiast możliwość zapisu małej ilości danych. kodów kreskowych oraz częsty problem z odczytem danych (dane trzeba wprowadzać ręcznie). Duży problem w wielu sytuacjach stanowi również niewielka dopuszczalna odległość kodu od czytnika. 
Technologia kodów kreskowych jest obecnie najpowszechniejszą i najtańszą technologią znakowania produktów, jednak osiągnęła już swoje granice możliwości – w pewnych zastosowaniach jest już niewystarczająca.
Coraz większą popularność zyskuje technika identyfikacji radiowej – \emph{RFID}. System jest bardzo prosty dzięki temu zyskuje na popularności zarówno przy dużych i kosztownych projektach jak również  niewielkich przedsięwzięciach.

\section{Struktura systemu \emph{RFID}}

RFID (ang.\emph {Radio Frequency  Identyfication}) to jedna z najszybciej rozwijających  się automatycznych technologii bezprzewodowej  identyfikacji. Technologia ta umożliwia zdalne przechowywanie i odczytywanie danych z układów nazywanych znacznikami (ang. \emph{tag}) za pośrednictwiem fal radiowych. 

\noindent  
\newline W skład typowego systemu \emph{RFID} wchodzą następujące podzespoły:
\begin{itemize}\setlength{\itemsep}{0pt}

	\item znacznik  (etykieta, transponder, tag) –zbudowany z układu elektronicznego najczęściej microchipa z pamięcią, w którym dane są  kodowane i zapisywane w pamięci podręcznej oraz transmitowane za pomocą anteny nadawczo odbiorczej (często napylanej na warstwie izolatora). Układ identyfikatora wykonany jest zazwyczaj na podłożu z papieru lub plastiku. 

	Etykiety \emph{RFID} mogą zawierać zarówno informacje zapisane w formie elektronicznej (pod postacią danych) jak również naniesiony tekst lub kod kreskowy. 

	Identyfikatory możemy sklasyfikować na dwa sposoby. W zależności od sposobu zasilania i sposobu zapisu danych. 

	\item czytnik - zbudowany z mikrokomputera, który weryfikuje poprawność otrzymanych informacji, modułu transmisji radiowej czyli nadajnika i odbiornika  odpowiedzialnych za odczyt i zapis danych w identyfikatorze oraz antenę lub cewkę. Niektóre czytniki posiadają dodatkowo interfejs łączący z komputerem \emph{PC} (pozwalający na transmisję danych pomiędzy czytnikiem a komputerem).
	
	Oprogramowanie czytnika składa się z warstwy komunikacyjnej i użytkowej. Warstwa komunikacyjna odpowiedzialna jest za techniczną stronę transmisji danych, a warstwa użytkowa odpowiada za poprawne działanie aplikacji czyli wymianę, gromadzenia i przetwarzanie informacji na serwerze lub aplikacji klienckiej.


\end{itemize}



\section{Kryteria podziału znaczników i czytników}

\noindent 
Ze względu na sposób zasilania znaczniki dzielimy na: 
\begin{itemize}\setlength{\itemsep}{0pt}

	\item pasywne  (ang. \emph{passive}) (bez wewnętrznego zasilania) do zasilania  wykorzystują energie pola elektromagnetycznego, która wytwarzana jest przez czytnik, są najczęściej stosowanymi znacznikami spośród wszystkich dostępnych, 

	\item aktywne (ang. \emph{active}) (z własnym źródłem zasilania np. baterią), emitują większą moc sygnału transmitowanego z identyfikatora i transmisja trwa krócej niż w innych znacznikach. Stosuje się je przede wszystkim do identyfikacji pojazdów,
	
	\item pół-pasywne (ang. \emph{semi-passive}) łączą cechy zarówno transponderów aktywnych jak i pasywnych. Transpondery te wyposażone są w baterię, która zasila obwód elektroniczny. Zasięg odczytu sięga 100 m, w dużym stopniu jednak za leży od czułości odbiornika w czytniku.  

\end{itemize}

\noindent 
\newline Ze względu na sposób zapisu danych znaczniki dzielimy na:
\begin{itemize}\setlength{\itemsep}{0pt}

	\item znaczniki typu \emph{RO} (ang.\emph{read only}) - tylko odczyt – są programowane podczas produkcji i zawierają numer seryjny (mogą zawierać również inne dane, których nie można modyfikować), 

	\item znaczniki typu \emph{RW} (ang.\emph{read write}) - odczyt i zapis – dane mogą być wielokrotnie modyfikowane (pamięć może być podzielona na dwie części: część tylko do odczytu i część w której użytkownik może modyfikować zapisane dane),

	\item znaczniki typu\emph{WORM} (ang.\emph{write once read many}) - jednokrotny zapis i wielokrotny odczyt, część danych zapisana jest trwale, a część może być zmieniona przez użytkownika.

\end{itemize}


\noindent 
\newline Klasyfikacji czytników można dokonać według dwóch kryteriów:

\begin{itemize}\setlength{\itemsep}{0pt}

	\item ze względu na na kryterium mobilności czytnika: 

	\item stacjonarny - pracujące w trybie autonomicznym – cały czas sczytuje etykiety znajdujące się w jego zasięgu, 

	\item interaktywny – w tym trybie komunikacja odbywa się za pomocą aplikacji pracującej na serwerze lub aplikacji klienckiej, 

	\item ruchome, przenośne.


	\begin{itemize}\setlength{\itemsep}{0pt}
		
		\item ze względu na typu interfejsu komunikacyjnego:

		\item szeregowe - porty RS-232 lub RS-485, 

		\item sieciowe z interfejsem sieciowym np. Ethernet.
	\end{itemize}
\end{itemize}

Technologia \emph{RFID} w ostatnim dziesięcioleciu zyskała miano jednego z najbardziej obiecujących i najprężniej rozwijających się systemów identyfikacji.
System \emph{RFID} pozwala na dużą automatyzację pracy zarówno podczas zapisu jak i odczytu danych. W przeciwieństwie do kodów kreskowych technologia ta nie wymaga bezpośredniej ,"widoczności" pomiędzy etykietą, a czytnikiem oraz daje możliwość odczytu i przetwarzania wielu etykiet równocześnie.

%\newpage

\section{Zakresy częstotliwości}

Obecnie wdrażane systemy \emph{RFID} pracują na różnych zakresach częstotliwości, które można podzielić na następujące pasma:

\begin{itemize}\setlength{\itemsep}{0pt}
	\item pasmo niskich częstotliwości \emph {LF} (125-134.2 kHz).
Znaczniki \emph{LF} to zazwyczaj tagi pasywne, które wyróżniają się, małą wrażliwością na obecność w pobliżu elementów metalowych, substancji płynnych oraz innych elementów wykonanych z dobrych przewodników. Dlatego świetnie sprawdzają się jako identyfikatory narzędzi, podzespołów maszyn, pojazdów czy metalowych kontenerów. Znalazły również zastosowanie w rolnictwie do oznakowania zwierząt, leków oraz produktów spożywczych, gdyż pole o niskiej częstotliwości przenika przez tkanki ciała i płyny.
Wygląd i budowa znaczników \emph{LF} zależy ściśle od ichprzeznaczenia. Znaczniki używane jako immobilizery (czyli elektroniczne zabezpieczenia przed niepowołanym uruchomieniem pojazdu) zazwyczaj wbudowane są w kluczyk natomiast cewka /  antena czytnika \emph{RFID} umieszczona jest współosiowo w stacyjce samochodu.
   
	\begin{figure}[h!]
	\centering
	    \includegraphics[width=9.32cm]{immobilizer.jpg}
	    \caption{Immobilizer}
	\end{figure}

	Dodatkowo systemy pracujące na częstotliwośći 135 kHz znalazł zastosowanie w automatycznej identyfikacji zwierząt hodowlanych i domowych.

	\begin{figure}[h!]
	\centering
	    \includegraphics[width=7.27cm]{automatyczna_identyfikacja.jpg}
	    \caption{System automatycznej identyfikacji}
	\end{figure}
	
	\begin{figure}[h!]
	\centering
	    \includegraphics[width=14.34cm]{znakowanie_zwierzat.jpg}
	    \caption{Mikroczipy i kolczyki stosowane do znakowania zwierząt oraz kapsułki (biochipy) do wszczepiania pod skórę}
	\end{figure}
	
	Znaczniki LF zbudowane są z ferrytowego rdzenia, na którym nawinięte są zwoje cewki. Występują w formie plastikowych kart, krążków, biochipów, pastylek. Wadami systemu \emph{RFID-LF} są: mała szybkość transmisji danych (1-2 kb/sek) - co znacznie ogranicza możliwą do zapisania / odczytania ilość danych, niewielki zasięg – około 0.5 metra, podatność na zakłócenia przez urządzenia elektryczne - co ogranicza zastosowanie w przemyśle, limit jednoczesnego odczytu nie więcej niż 20 transponderów – ogranicza to pojemność systemu i maksymalną liczbę odczytów jaką może obsłużyć jeden czytnik. 

	\item pasmo wysokich częstotliwości \emph{HF} (częstotliwości to 6.78 MHz, 13.56 MHz, 27.125 MHz oraz 40.68 MHz). 
Transpondery \emph{HF} pracujące w zakresie 13.553 MHz – 13.657 MHz to zazwyczaj tagi pasywne. Identyfikatory pracujące na wysokich częstotliwościach wykazują się większę wrażliwością na obecność w swoim otoczeniu metali i mniejszym wpływem na zakłócenia elektromagnetyczne pochodzące od innych urządzeń niż tagi \emph{LF}. Pamięć tagów \emph{HF} ma kilka razy większę pojemność, a szybkość komunikacji sięga 20 kbit/s. Cewka transpondera \emph{HF} zbudowana jest zazwyczaj z 3-8 zwojów, nadrukowana jest za pomocą przewodzącego lakieru na podłoże i zaprasowana w etykiecie. Znaczniki te produkowane są w postaci samoprzylepnych etykiet (ang.\emph{smart label}). Zaletami transponderów \emph{HF} są znacząco mniejsze koszty wykonania w porównaniu z identyfikatorami LH oraz umożliwia to niewielka grubość identyfikatora nawet 0,1 mm. System RFID-HF daje możliwość odczytu do 50 znaczników jednocześnie (dzięki wprowadzeniu mechanizmów antykolizyjnych) co pozwala zastosować je do automatycznej identyfikacji produktów i obiektów. System wymaga zachowania odległości co najmniej 2-3 centymetrów pomiędzy transponderami. 
	
	Typowo zasięg działania systepu RFID pracującego w paśmie HF nie przekrasza 1,5 metra. Można uznać to za zaletę w przypadku systemu typu PayPass (płatności zbliżeniowe). Właściwość ta dodatkowo podnosi poziom bezpieczeństwa całego systemu, zmniejszając liczbę prób nieautoryzowanego dotępu.  

	\begin{figure}[h!]
	\centering
	    \includegraphics[width=16.56cm]{karty_platnicze.jpg}
	    \caption{Karty i breloki systemu PayPass}
	\end{figure}

	Systemy \emph{HF} są popularne ponieważ posiadają możliwość wielokrotnego zapisu danych, która jest niezbędna zarówno przy znakowaniu książek, dokumentów (paszportów, legitymacji studenckich, biletów komunikacji miejskiej, kart płatniczych) jak i bagażu na lotnisku czy odzieży w pralni.

	\begin{figure}[h!]
	\centering
	    \includegraphics[width=15.51cm]{bagaz.jpg}
	    \caption{Znakowanie bagażu lotniczego}
	\end{figure}

	\begin{figure}[h!]
	\centering
	    \includegraphics[width=16.82cm]{dokumenty.jpg}
	    \caption{Znakowanie książek, dokumentów, kart miejskich}
	\end{figure}
	
	 System znaczników pracujących na wysokich częstotliwościach znalazł zastosowanie w Europejskim Systemie Sterowania Pociągiem (ang. \emph{ETCS - European Train Control System}). System ten wyposażony w jest w urządzenie zwane Eurobalisą, które mocowane jest na torze pomiędzy szynami i może komunikować się z przejeżdżającymi nad nim pociągami.


	\item pasmo ultra wysokich częstotliwości \emph{UHF} (częstotliwości 433.92 MHz i 860-960 MHz).
Zakresy częstotliwości \emph{UHF} podzielono na kilka podzakresów np. Europa 860-868 MHz, Ameryka Północna 902-928 MHz, Japonia  950 MHz-956MHz. W związku z różnicami w częstotliwościach pracy urządzeń, ustalono globalny standard ISO/IEC 18000-6 C. W celu objęcia globalnego łańcucha dostaw stworzono identyfikatory i czytniki, która mogą pracować na całym świecie. Znaczniki pasywne systemów \emph{RFID} pracujących w pasmie \emph{UHF} pozwalają na dużo większy zasięg niż znaczniki pasywne pozostałych pasm częstotliwości. Zaimplementowano lepsze protokoły antykolizyjne niż w systemach \emph{HF}. Wskutek tego zwiększyła się możliwość odczytu do 200 znacznikóa jednocześnie. Zasięg odczytu zawiera się w granicach od 3 do 6 metrów.Dodatkowo duża szybkość transmisji danych do 120 kbit/sek, powoduje, żę znajdują one zastosowanie w magazynach.

Podobnie jak znaczniki \emph{HF} nie mogą być używane w pobliżu elementów metalowych, substancji płynnych i innych dobrych przewodników. 
Znaczniki \emph{UHF} ze względu na duży zasięg bardzo często stosowane są do śledzenia obiektów w logistyce. Wiele firm logistycznych, handlowych i spedycyjnych używa tylko i wyłącznie znaczników pasywnych \emph{UHF} ze względu na bardzo niski koszt produkcji, wynoszący kilkanaście groszy. 
Częstotliwość \emph{UHF} jest najbardziej rozpowszechnioną częstotliwością w logistyce i produkcji.

	\begin{figure}[h!]
	\centering
	    \includegraphics[width=14.37cm]{logistyka_transport_przemysl.jpg}
	    \caption{Przykłady zastosowań tagów UHF w sektorze logistyki, produkcji, transportu i handlu}
	\end{figure}

	\item pasmo mikrofalowe \emph{MW} (częstotliwości to 2.45 GHz, 5.8 GHZ i 24.125 GHz).
W paśmie \emph{MW} wykorzystuje sie w większości znaczniki aktywne lub pasywno-aktywne. Są  mniejsze niż tagi pracujące na innych pasmach częstotliwości jednak droższe. Zasięg odczytu dochodzi do kilkuset metrów. Najistotniejszą zaletą tagów pracujących na częstotliwości mikrofalowej okazuje się wysoki transfer danych, który  umożliwił odczyt informacji z obiektów poruszających się z prędkością ponad 100 km/h, co nie jest możliwe w technologii tagów \emph{LH} i \emph{HF}. Pozwala to na wykorzystanie ich do identyfikacji środków transportu komunikacji miejskiej, samochodów poruszających się  na autostradzie (automatyczne naliczanie opłaty za przejazd autostradą), rejestracji przejeżdżających pociągów, czy statków.

	\begin{figure}[h!]
	\centering
	    \includegraphics[width=16.61cm]{autostrady.jpg}
	    \caption{Naliczanie opłat za przejazd autostradą oraz identyfikacja środków komunikacji miejskiej}
	\end{figure}

	Znaczniki pracujące na częstotliwości 5.8 GHz montowane są w czujnikach przed drzwiami (w sklepach i marketach), w czujnikach ruchu jak również w systemach automatycznego spłukiwania toalet.
	Jedną z wad znaczników \emph{MW} jest możliwość wystąpienia zjawiska interferencji, które zakłóci  transmisję w środowisku o dużym zagęszczeniu znaczników jak również spowoduje zakłócenia w działaniu tych pracujących  w pobliżu substancji płynnych lub metali.

\end{itemize}

\newpage

\section{Główne obszary zastosowania systemów \emph{RFID}}

\noindent 
Technologia \emph{RFID} znalazła zastosowanie w niżej wymienionych sektorach gospodarki: 

\begin{itemize}\setlength{\itemsep}{0pt}
	\item logistyka, transport i magazynowanie– gdy zachodzi potrzeba szybkiej identyfikacji i śledzenia poszczególnych elementów łańcucha dostaw. Identyfikacja i rejestracja towarów w magazynie i na kolejnych etapach produkcji / dystrybucji.  Śledzenie przesyłek pocztowych / kurierskich, bagaży na lotnisku.  Znaczniki \emph{RFID} stopniowo wypierają etykiety logistyczne i kody kreskowe,

	\item płatności elektroniczne– systemy płatnicze PayPass, systemy kart rabatowych w hipermarketach i na stacjach benzynowych, 

	\item elektroniczne bilety, karty wstępu – komunikacja miejska, imprezy masowe/imprezy sportowe, opłaty za przejazd autostradami, systemy \emph{skipass},

	\item organizacja pracy – kontrola dostępu do wyznaczonych miejsc poprzez umieszczanie znaczników \emph{RFID} w identyfikatorach (np. kartach, breloczkach). Pozwalają one na identyfikację właściciela, monitoring czasu pracy oraz system kontroli dostępu (inteligentne budynki),

	\item znakowanie zbiorów, odzieży – archiwa, muzea i biblioteki w których na książkach coraz częściej zamiast kodów kreskowych znajdują się tagi RFID, które ułatwiają identyfikacje jak również zabezpieczają przed kradzieżą. Znakowanie odzieży w celu rozpoznawania i sortowania m.in. pralniach jak i w sklepach również w celu ochrony przed kradzieżą jak i fałszerstwem produktów markowych,
 	
 	\item dokumenty, papiery wartościowe– znaczniki w postaci samoprzylepnych etykiet. Śledzenie i zapisywanie historii obiegu oraz aktualnej lokalizacji ważnych dokumentów.  W paszportach stosowane w celu przechowywania danych osobowych oraz informacji na temat przekraczanych granic. Wymagane zabezpieczenie przed odczytem danych przez niepowołane osoby,
	
	\item przemysł – oznaczanie i  identyfikacja podzespołów, półproduktów oraz zapis poszczególnych stanów procesów produkcyjnych, co pozwala na ich automatyzację. Śledzenie obiektów na liniach produkcyjnych, rejestracja danych. Identyfikacja cystern, wagonów pojemników w procesie produkcji
	
	\item dane eksploatacyjne maszyn – identyfikacja tonerów do drukarek tak, aby drukarki nie podejmowały pracy w momencie przekroczenia dopuszczalnej liczby wydrukowanych stron na jednym tonerze lub w przypadku gdy toner nie pochodzi do konkretnego producenta, 
	
	\item rolnictwo, znakowanie zwierząt hodowlanych, zagrożonych gatunków – oznakowanie zwierząt za pomocą znaczników występujących w postaci charakterystycznych żółtych klipsów przymocowywanych do uszu bydła i trzody chlewnej. Pozwalają na identyfikację zwierząt a co za tym idzie odczytanie m.in. miejsca ich chowu,
	
	\item sport – rejestrowanie czasów oraz ilości okrążeń w zawodach sportowych.  W przypadku masowych rozgrywek sportowych  nie możliwe byłoby odmierzanie czasu stoperem każdemu zawodnikowi,
	
	\item transpondery na oponach samochodowych – zawierają najczęściej numer identyfikacyjny, datę produkcji oraz stopień zużycia opony.

\end{itemize}


\section{Wykorzystanie technologii \emph{RFID} w różnych sektorach gospodarki}

	\begin{figure}[h!]
	\centering
	    \includegraphics[width=16.2cm]{wykres_ver3.jpg}
	    \caption{Procentowy udział branż wykorzystujących technologię \emph{RFID}}
	\end{figure}


\section {Zalety i wady zastosowania systemu \emph{RFID}}

\noindent 
Zalety technologii \emph{RFID}:

\begin{itemize}\setlength{\itemsep}{0pt}
	\item nie jest wymagany optyczny kontakt zarówno przy zapisie jak i odczycie, czytnika z identyfikatorem, dzięki temu operację ta można zautomatyzować i zrealizować szybciej,
	
	\item możliwość odczytu wielu identyfikatorów równocześnie,
	
	\item duża pojemność pamięci, możliwość przechowywania większej ilości danych niż w przypadku kodów kreskowych

	\item możliwość aktualizacji / zmiany, wielokrotnego zapisywania i dopisywania danych  na etykiecie,

	\item identyfikatory mogą być wykorzystane wielokrotnie,

	\item duża prędkość transmisji danych,

	\item możliwość zapisywania danych w trakcie ruchu obiektu,

	\item duże bezpieczeństwo danych -istnieje możliwość zastosowana różnych metod ochrony danych zapisanych na znaczniku, zabezpieczenie dostępu do pamięci, zabezpieczenie dostępu hasłem,  

	\item miniaturyzacja znacznika - najmniejszy wyprodukowany na świecie znacznik przez firmę Hitachi posiada wymiary 0,05 x 0,05 mm

	\item możliwość pracy w trudnych warunkach przemysłowych w miejscach, gdzie występuje duże zapylenie, bardzo wysokie lub bardzo niskie ciśnienie, niska temperatura, agresywne środowisko chemiczne np. kopalnie, zakłady przemysłowe, chłodnie itp.

	\item możliwość integracji z istniejącymi systemami automatycznej identyfikacji (kody kreskowe),

	\item możliwość wykorzystania tych samych identyfikatorów w całym łańcuchu dostaw.

\end{itemize}

\noindent 
\newline Wady technologii \emph{RFID}:


\begin{itemize}\setlength{\itemsep}{0pt}

	\item brak jednolitego standardu dla protokołu RFID,
	
	\item możliwość zakłócenia przez odziaływanie elektromagnetyczne, wilgoć czy metale,
 	
 	\item większy koszt pojedynczego znacznika, w porównaniu z kodem paskowym, zwłaszcza w przypadku znaczników o większej funkcjonalności 

\end{itemize}





\chapter{Technologia anten konforemnych}

Układy antenowe o elastycznych, giętkich podłożach są trudne technologicznie do wykonania jednak mają szerokie zastosowanie praktyczne ze względu na możliwosć kształtwoania powierzchni. 
W trakcie realizacji konforemne anteny i układy antenowe sprawiają wiele problemów konstrukcyjnych, gdyż nawet małe deformacje i odkształcenia mają duży wpływ na parametry polowe anten. 
Anteny konforemne dają możliwość uzyskania szerszego pokrycia kątowego płaszczyzny skanowania. Celem uzyskania poprawnych charakterystyk promieniowania, należy w trakcie projektowania uwzględnić zmiany amplitud i faz sygnałów otrzymanych na poszczególnych elementach układu antenowego, które są wynikiem elastycznej struktury.    
Metody projektowania i konstruowania anten są na tyle skomplikowane, że cały proces łącznie z badaniami, aż do praktycznej realizacji przebiega z udziałem grupy sapecjalistów z różnych dziedzin (kompatybilności elektromagnetycznej, technologii materiałowych, systemów tekstronicznych, radioelektroniki, telekomunikacji, przetwarzania sygnałów i innych). Swoistość anten o elastycznej strukturze predysponuje je głownie do wykorzystania w pasmie powyżej 1 GHz.     

\section{Obszary wykorzystania i zastosowania anten konforemnych}

Możliwości zastosowania tego typu anten jest wiele. Dziedziny zastosowań wzajemnie się przenikają i uzupełniają, jednak można wyróżnić główne trzy, do których należą:

\begin{itemize}\setlength{\itemsep}{0pt}
	
	\item zastosowania medyczne - radary przenzaczone do mocowania na skórze: sportowców, ratowników medycznych, strażaków, żołnierzy. Czujniki służą do monitorowanie aktywności (rytmu serca, oddechu) i stanu fizjologicznego człowieka (poziomu wydzielanego potu, stężenie tlenu i dwutlenku węgla we krwi, stanu odwodnienia, poziomu elektrolitów).Przekazywanie informacji o parametrach życiowych (np.temperaturze ciała). Wykrywanie zewnętrznych zagrożeń np. stężenie niebezpiecznych substancji chemicznych, gazów, oparów. Możliwość zaastosowana do wykrywania komórek nowotworowych,  

\begin{figure}[h!]
\centering
	\includegraphics[width=14cm]{anteny_konforemne.jpg}
	\caption{Przykłady zastosowań anten konforemnych}
\end{figure}

	\item zastosowania militarne - system lokalnych sieci WBAN, wspieranie systemów rozpoznawania terenu UAVs (mobilny system łącznosci), określanie lokalizacji np. żołnierza, maskowanie aparatury do łączności.
	Zastapienie kilku urządzeń monitorujących, systemem anten, a więc zmniejszenie wagi potrzebnego sprzętu. Zwiększenie niezawodności łączności (z ratownikami medycznymi, żołnierzami), a co za tym idzie poprawa mobilności służb,      

	\item zastosowanie to tworzenia krótkodystansowych systemów łączności bezprzewodowej - zastsowanie w sieciach: \emph{WPAN} (ang.\emph{Wireless Personal Area Network}), \emph{WBAN} (ang. \emph{Wireless Body Area Network}) oraz \emph{UWB} (ang. \emph{Ultra WideBand}). Współpraca czujników wszytych w ubranie (monitorowanie oznak życia organizmu). Integracja anten z elementami wyposażenia: kamizelką, rękawem, kołnierzem, hełmem (np. strażaka, żołnierza) przy użyciu BLuetooth, GSM lub UMTS. Kominikacja anten z urzadzeniami działającymi w sieciach Wireless USB, Bluetooth, WLAN.  

\end{itemize}
	

\section{Częstotliwosci pracy dla anten konforemnych}
 
\noindent Pasma pracy:

\begin{itemize}\setlength{\itemsep}{0pt}
	
	\item UWB/WUSB: 3.1GHz - 10.6 GHz (FCC) - pasmo stosowane w systemach łączności krótkodystansowej, bazujące na przesyłaniu ultra-krótkich impulsów oraz charakteryzujące się niskim poborem mocy,

	\item ISM: 2.4 GHz - 2.4835 GHz - zakres używany przede wszystkim dla systemów łączności krótko i średniodystansowych takich jak Bluetooth, sieci WLAN / WBAN / WPAM,

	\item GPS: L1: 1575.42 MHz, L2: 1227.6 MHz, Galileo: 1176.45 MHz - częstotliwość pracy wykorzystywana do dla nawigacji satelitarnych GPS, Galileo i innych oraz do współpracy Galileo z innymi systemami takim jak GPS, Glonass, Loran-C, UMTS.

	\item ZigBee: 868 MHz, 915 MHz, 2.4 GHz - częstotliwości te znalazły zastosowanie w tworzeniu sieci czujników łączących urządzenia techniczne.

	\item Zastosowania militarne: 50-300 MHz pasamo pracy przeznaczone dla radarów przenośnych oraz kamizelek z zintegrowanymi systemami łączności (antenami).

\end{itemize}

\section{Anteny tekstylne}

Tekstronika jest nową gałęzią nauki, zajmującą się projektowaniem i budową inteligentnej odzieży. Jest wynkiem połączenia kilku dziedzin nauki m.in. nauki o materiałach, włókiennictwa, elektroniki, radiokumunikacji, kompatybilności elektromagnetycznej jak również informatyki. Tekstronika stwarza możliwość wykorzystania różnych metod transmisji sygnałów (rys.2.2) w układach antenowych. W swojej pracy skoncentruję się na transmisji bezprzewodowej w paśmie nielicencjonowanym 2.4 GHz - 2,4835 GHz.  
Anteny do zastosowań tekstronicznych, czyli implementowane w odzieży, dzięki eleastycznej i giętkiej konstrukcji są łatwe do scalenia z materiałami.


\begin{figure}[h!]
	\centering
	    \includegraphics[width=15.5cm]{diagram.jpg}
	    \caption{Podział tekstronicznych systemów transmisji}
\end{figure}


 \chapter{Cel i zakres pracy}

Celem niniejszej pracy było zaprojektowanie i wykonanie anteny tekstylnej przewidzianej do pracy w paśmie \emph{ISM} (2.4 GHz). 
Praca nad projektem inżynierskim przebiegała w kilku etapach:

\begin{itemize}\setlength{\itemsep}{0pt}
	
	\item stworzenie modelu symulacyjnego badanej anteny,

	\item symulacje zaproponowanej struktury anteny,

	\item przeprowadzenie analizy numerycznej,

	\item wykonanie modelu i przeprowadzenie badań eksperymentalnych wybranych parametrów anteny.

\end{itemize}

W trakcie realizacji założeń projektu pracowałam wykorzystując środowisko \emph{CST Microwave Studio}. Program umożliwia zaprojektowanie i dokładną symulację anteny.   



\chapter {Projekt anteny}

\section{Wstęp}

\noindent Ważniejsze parametry anten istotne w procesie projektowania: 

\begin{itemize}\setlength{\itemsep}{0pt}

	\item pasmo pracy - czyli zakres częstotliwości, dla których projektowana antena spełnia założone wcześniej kryteria. Pasmo pracy anteny powinno być możliwie szerokie - dla anten RFID nie powinno jednak przekraczać 30 MHz,

	\item ipedancja wejściowa - jest to stosunek napięcia do natężenia prądu na zaciskach wejściowych anteny.W przypadku gdy mamy do czynienia z dostrojeniem rezonansowym do częstotliwości pracy, to impedancja wejściowa jest czystą rezystancją - w innych wypadkach pojawia się także reaktancja.
	standardowa impedancja linii transmisyjnych wynosi 50\(\Omega\). Wartość ta pozwala uzyskać maksymalną mocy sygnału, przy minimalnych stratach w linii,

	\item zysk energetyczny - defionowany jako zysk kierunkowy zględem anteny wzorcowej (anteny izotropowej).
	Jest to stosunek gęstości mocy promieniowania w danym kierunku do średniej gęstości mocy wypromieniowanej przez antenę w pełnym kącie bryłowym. Podawany zazwyczaj dla kierunku, w którym promieniwanie anteny jest najsilniejsze, 

	\item charakterystyka promieniowania - pokazuje w jaki sposób antena promieniuje energię w różnych kierunkach. Obrazuje unormowany rozkład pola elektrycznego. Charakterystyka może być wyznaczona w dwóch płaszczyznach lub mieć postać trójwymiarową,

	\item polaryzacja anteny - określa ją polaryzacja wytwarzanej przez nią fali elektromagnetcznej 
	Wyróżniamy polaryzację liniową (pionową, poziomą, nachyloną pod kątem), eliptyczną lub kołową ( lewoskrętną, prawoskrętną),

	\item sprawność - to storunek mocy wypromieniowanej przez antenę do mocy dostarczonej do anteny,

	\item dopasowasowanie - linia transmisyjna jest dopasowana gdy impedancja charakterystyczna linii jest taka sama jak anteny. 

\end{itemize}

\newpage
\section{Struktura}

Do realizacji projektu anteny tekstylnej RFID wykorzystano środowisko \emph{CST Microwave Studio}. Inspiracją do stworzenia struktury wyjściowej był artykuł z 6th European Conference on Antennas and Propagation \cite{Artykul}. Strukturę bazową poddano modyfikacji celem dostosowania jej do założeń projektu. Podłoże wykonano z papieru (celem bezproblemowego przymocowania anteny do ramienia). Optymalizację parametrów symulacji, przeprowadzono poleceniem \emph{adaptive mesh refinement}, które ma na celu dostosowanie rozmiaru siatki, dzięki czemu otrzymane wyniki są dokładniejsze. 


\begin{figure}[h!]
	\centering
	    \includegraphics[width=15.5cm]{struktura.jpg}
	    \caption{Model struktury anteny}
\end{figure}


\begin{figure}[h!]
	\centering
	    \includegraphics[width=15.5cm]{fizyczna_antena.jpg}
	    \caption{Prototyp anteny}
\end{figure}

\begin{table}[h]
\begin{center}
    \begin{tabular}{cccc}
    PARAMETR           & WARTOŚĆ \\ \hline
    ~                  & ~       \\
    L                  & 130 mm  \\
    ~                  & ~       \\
    L1                 & 55.5 mm \\
    ~                  & ~       \\
    L2                 & 59 mm   \\
    ~                  & ~       \\
    H                  & 30 mm   \\
    ~                  & ~       \\
    H2                 & 22 mm   \\
    ~                  & ~       \\
    D                  & 5 mm    \\
    ~                  & ~       \\
    D2                 & 2 mm    \\
    ~                  & ~       \\
    W                  & 3 mm    \\
    ~                  & ~       \\
    W2                 & 17 mm   \\
    ~                  & ~       \\
    DLUGOSC PODLOZA    & 15.5 mm \\
    ~                  & ~       \\
    SZEROKOSC PODLOZA  & 7.5 mm  \\
    ~                  & ~       \\
    \end{tabular}
 \end{center}
\end{table}


Przeprowadzono parametryzację struktury, aby uzyskać założone parametry anteny. Stopień dyskretyzacji został ustalony po 6 iteracjach. Rezultaty procesu doboru wymiarów anteny, można było obserwować na wykresach. Najlepsze wyniki posłużyły do ustalenia rozmiaru projektowanej anten, a pózniej wykonania prototypu. Schemat struktury i prototyp widoczny jest na Rys.4.1 i Rys. 4.2. Natomiast wymiary zestawiono w Tab. 4.1. 
\newline
\noindent
Wykonana antena ma strukturę dipola o polaryzacji poziomej. 
W skład układu antenowego wchodzi część promieniująca zwana promiennikiem lub też radiatorem oraz część zasilająca. Promiennik ma za zadanie wypromieniować w przestrzeń energię dostarczoną przez kabel zasilający.



%Zmieniająć długość L - (długość ramion promiennika) możem
%Impedancja wejściowa symulowanej anteny równa jest 38 \(\Omega\). 

%W trakcie pomarów Na kabel zasilający antenę zostały nałożone koraliki ferrytowe, aby zmniejszyć do minimum promieniowanie kabla i ewentualne zakłócenia z tym związane. 


%Aby uzyskać pracę w rezonansie należy wówczas odpowiednio skrócić długość anteny. 



%Zadaniem części promieniującej, zwanej promiennikiem lub radiatorem, jest wypromieniowanie w przestrzeń dostarczonej do niego energii w.cz. Promiennik charakteryzuje się zakresem częstotliwości, impedancją wejściową, polaryzacją, współczynnikiem kierunkowości, zyskiem oraz wymiarami.
% Na kabel zasilający antenę zostały nałożone koraliki ferrytowe, aby zmniejszyć do minimum promieniowanie kabla i ewentualne zakłócenia z tym związane. 
%Wraz ze zmnianą długości ramion promiennika dobrana została częstotliwość na poziomie 2.4 GHz.      





%Wpływ człowieka na charakterystykę promieniowania

 



%Obecność człowieka pogarsza charakterystykę promieniowania anteny. Zjawisko to spowodowane jest przewodzącymi właściwościami ciała człowieka, co powoduje tłumienie składowej elektrycznej pola. Na zakłócenia tego typu bardziej odporne są anteny pętlowe. Z tego powodu częściej znajdują one zastosowanie w urządzeniach przenośnych (np. pilotach zdalnego sterowania), na które mogłaby wpływać bezpośrednia bliskość człowieka.

 
%Tworzenie modelu symulacyjnego  Modelowanie      

\chapter {Wyniki pomiarów}

\begin{table} [h]
\begin{center}
    \begin{tabular}{c|cc}
    RODZAJ POLARYZACJI  & ANTENA WZORCOWA & ANTENA MIERZONA TEKSTYLNA \\ \hline
    ~                   & ~               & ~                         \\
    POLARYZACJA POZIOMA & -34.5 dBm       & -39,2 dBm                 \\
    ~                   & ~               & ~                         \\
    POLARYZACJA PIONOWA & -59.7 dBm       & -59.9 dBm                 \\
    ~                   & ~               & ~                         \\
    \end{tabular}
 \end{center}
\end{table}






\begin{table}[h]
\begin{center}
    \begin{tabular}{|c|c|c|c|}
    \hline
    ~               & \multicolumn{3}{c|}{~}     \\
    ~               & \multicolumn{3}{c|}{ANTENA MIERZONA TEKSTYLNA } \\
    ~               & \multicolumn{3}{c|}{~}     \\ \hline
    ~               & \multicolumn{2}{c|}{~}     & ~                                  \\
    ANTENA WZORCOWA & \multicolumn{2}{c|}{BEZ OBECNOSCI CZLOWIEKA} & W OBECNOSCI CZLOWIEKA              \\
    ~               & \multicolumn{2}{c|}{~}     & ~                                  \\ \hline
    ~               & ~                          & ~                                  & ~                                  \\
    ~               & POLOZENIE ----             & POLOZENIE [     ]                  & POLOZENIE [    ]                   \\
    ~               & ~                          & ~                                  & ~                                  \\ \hline
    ~               & ~                          & ~                                  & ~                                  \\
    -10.24 dB       & -38.25 dB                  & -46.05 dB                          & -49.04 dB                          \\
    ~               & ~                          & ~                                  & ~                                  \\ \hline
    \end{tabular}
\end{center}
\end{table}






%Anteny tekstylne to anteny pracujące w pobliżu ciała ludzkiego, a co za tym idzie patrząc od strony technologicznej - w pobliżu dielektrycznego, dyspersyjnego ośrodka startnego. Fale o takiej samej częstotliwośći (w ośrodku dyspersyjnym), rozchodzą się różną prędkością. 
%Ponaddto tkanka ludzka jest ośrodkiem stratnym o skończonej konduktywności.



%\begin{center}
%   \begin{tabular}{ | l | p{2.5cm} | p{2.5cm} | p{2.5cm} | p{2.5cm} |}
%    \hline
%        Smartphone OS & aaaa & bbbb & ccccc & dddd \\ \hline 
%      Android 2.2 & 299.94 & 509.32 & 164.76 & 537.95 \\ \hline
%       Android 2.3 & 133.50 & 154.76 & 57.65 & 215.13 \\ \hline
%        Android 4.0 & 97.05 & 103.44 & 31.80 & 179.36 \\ \hline
%        Android 4.1 & 75.60 & 85.57 & 21.60 & 132.03 \\ \hline
%       Android 4.2 & 55.28 & 83.76 & 23.28 & 121.51 \\ \hline
%       iPhone iOS5 & 21.61 & 132.90 & 8.64 & 177.69 \\ \hline
%        iPhone iOS6 & 15.71 & 73.86 & 7.27 & 109.32 \\ \hline
%    \hline
%    \end{tabular}
%\end{center}

%W poniżsyzm rozdziale 









	
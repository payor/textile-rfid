%##############################################################################
% Preambu³a dokumentu
% 17.12.2010
%
\documentclass[12pt,a4paper,twoside
%,draft%<------------------- opcja "draft" tylko markuje rysunki w tekscie
]{report}
\usepackage{graphicx}
\usepackage[pdftex]{hyperref}
\usepackage{breakurl}
\usepackage{polski}
\usepackage[utf8]{inputenc}
%\usepackage[cp1250]{inputenc}%<------------ strona kodowa dla MSWindows
\usepackage{t1enc,amsmath}
\usepackage{times,caption}
\usepackage{bsc_pl}
\usepackage{fancyhdr}%<-------------------- formatowanie nag³ówka i stopki
\usepackage{float}

% ustawienia obrazow %
\graphicspath{ {./assets/} }
\DeclareGraphicsExtensions{.pdf,.png,.jpg}

\linespread{1.5}

\pagestyle{fancy}%<--------------- styl z fancyhdr.sty; definicje nag³ówka
%-----------------------------------------------------------------------
% Definicje nag³ówka dostêpne z pakietem fancy.sty
\renewcommand{\headheight}{15pt}%<---------------- szerokoœæ nag³owka
\renewcommand{\headrulewidth}{0.4pt}%<------------ gruboœæ linii w nag³ówku
\fancyhead[RO,LE]{\thepage}
\fancyhead[LO]{\footnotesize \slshape \rightmark}%  <- mniejsze litery w tytule sekcji
\fancyhead[RE]{\footnotesize \slshape \leftmark}%   <-- i nag³ówka
\cfoot{}%<---------------------------------- likwidacja numerów w stopce
%-------------------------------------------------------------------------
\newcommand*{\mbf}{\mathbf}
\newcommand*{\bs}{\boldsymbol}
%
%\renewcommand*\captionlabeldelim{.}
\renewcommand{\captionlabelfont}{\bfseries}
\renewcommand{\captionsize}{\small}
%
% Koniec preambu³y
%##############################################################################
%
% Pocz¹tek w³aœciwego dokumentu
%
\begin{document}
%
\bibliographystyle{unsrt} % <- Ustawienie stylu bibliografi (unsrt,abbrv)
%<------------------------------------------ 
% Dane do strony tytu³owej
% Zmieniæ argumenty poleceñ: \author, \title i \graduationdate
%
\author{Imię i Nazwisko} 
\title{Tytuł pracy}
\graduationdate{Gliwice, styczeń 2013}
%<------------------------------------------
% Tworzona strona tytu³owa 
\titlepage 
\thispagestyle{empty}%<--------------------- pusta strona na odwrocie tutu³owej
\cleardoublepage%<-------------------------- dalej bêdzie od nieparzystej
%<------------------------------------------
% Tworzony spis treœci
\renewcommand*{\baselinestretch}{0.8}
\tableofcontents
\newpage
\thispagestyle{empty}
\cleardoublepage
\renewcommand*{\baselinestretch}{1.0}
%-------------------------------------------------------------------------------
% Wstawianie wykazu oznaczeñ
\addcontentsline{toc}{chapter}{Wykaz ważniejszych oznaczeń} % <-----  Je¿eli w 
% dokumencie nie ma wykazu oznaczeñ, nale¿y wykomentowaæ t¹ liniê
%\chapter*{Wykaz ważniejszych oznaczeñ}
{\small
\begin{tabular}{ccp{10cm}}
  $\bf{A}$ & -- & magnetyczny potencja³ wektorowy\\
  $AF$ & -- & charakterystyka grupowa dyskretnego uk³adu antenowego\\
  $A,~B$ & -- & funkcja\\
  $c$ & -- & prêdkoœæ œwiat³a w~pró¿ni\\
  $d$ & -- & wysokoœæ modu³u anteny, odleg³oœæ pomiêdzy Ÿród³ami modelu dyskretnego\\
  $D$ & -- & najwiêkszy wymiar liniowy anteny\\
  $\bf{D}$ & -- & wspó³czynnik dyfrakcji\\
  ${\bf E}$ & -- & zespolony wektor natê¿enia pola elektrycznego\\
  ${\bf E}_i$ & -- & sk³adowa indukcyjna pola elektrycznego\\
  ${\bf E}_p$ & -- & sk³adowa promieniowania pola elektrycznego\\
  $E$ & -- & wartoœæ skuteczna natê¿enia pola elektrycznego\\
  ${\bf E}^p$ & -- & sk³adowa elektryczna pola pierwotnego\\
  ${\bf E}^w$ & -- & sk³adowa elektryczna pola wtórnego\\
  $e$ & -- & podstawa logarytmu naturalnego \\
  $\bf{F}$ & -- & elektryczny potencja³ wektorowy\\
  $F$ & -- & ca³ka Fresnela\\
  $f$ & -- & czêstotliwoœæ, funkcja\\
  $[F],~[G]$ & -- & wektor kolumnowy\\
  F/B & -- & stosunek promieniowania przód/ty³\\
  $F_{\: \rm m}(\theta,\phi)$ & -- &
  charakterystyka promieniowania odosobnionego modu³u
  anteny dla pola elektrycznego\\
  $F_{\: \rm V}(\theta)$ & -- &
  charakterystyka promieniowania odosobnionego modu³u
  anteny w~p³a\-szczy\-Ÿnie pionowej\\
  $F_{\: \rm H}(\phi)$ & -- &
  charakterystyka promieniowania odosobnionego modu³u
  anteny w~p³a\-szczy\-Ÿnie poziomej\\
  $g$ & -- & funkcja\\
  $G(\phi, \theta)$ & -- & funkcja zysku energetycznego anteny\\
  $G$ & -- & zysk energetyczny\\
  $G_c(\phi)$ & -- & funkcja zysku energetycznego dla fali cylindrycznej\\
  $G_c$ & -- & zysk energetyczny Ÿród³a fali cylindrycznej\\
  $g_{\rm H}(\phi)$ & -- & charakterystyka promieniowania mocy w~p³a\-szczy\-Ÿnie poziomej\\
  $G_{\rm m}(\phi, \theta)$ & -- & funkcja zysku energetycznego odosobnionego modu³u anteny
      \end{tabular}
} \newpage  \thispagestyle{empty}
 {\small
\begin{tabular}{lcp{9cm}}
  $G_{\rm m}$ & -- & zysk energetyczny odosobnionego modu³u anteny\\
  $h$ & -- & wysokoϾ anteny\\
  $\bf{H}$ & -- & zespolony wektor natê¿enia pola magnetycznego\\
  $\bf{H}_i$ & -- & sk³adowa indukcyjna pola magnetycznego\\
  $\bf{H}_p$ & -- & sk³adowa promieniowania pola magnetycznego\\
  $H$ & -- & wartoœæ skuteczna natê¿enia pola magnetycznego\\
  ${\bf H}^p$ & -- & sk³adowa magnetyczna pola pierwotnego\\
  ${\bf H}^w$ & -- & sk³adowa magnetyczna pola wtórnego\\
  $[I]$ & -- & wektor pr¹du \\
  $j$ & -- & jednostka urojona \\
  $k$ & -- & liczba falowa \\
  ${\bf J}_s$ & -- & gêstoœæ elektrycznego pr¹du powierzchniowego\\
  $[L]$ & -- & macierz kwadratowa\\
  $L$ & -- & operator liniowy\\
  $l$ & -- & d³ugoœæ\\
  ${\bf M}_s$ & -- & gêstoœæ magnetycznego pr¹du powierzchniowego\\
  $P$ & -- & moc wypromieniowana przez antenê\\
  ${\bf r}$ & -- & wektor wodz¹cy \\
  $r_{\rm d}$ & -- & granica obszaru pola dalekiego odosobnionego
  modu³u anteny\\
  $R,~r$ & -- & odleg³oœæ\\
  $R$ & -- & wspó³czynnik odbicia \\
  $\bf{S}$ & -- & zespolony wektor gêstoœci strumienia mocy\\
  $S_c$ & -- & wartoœæ œrednia gêstoœci strumienia mocy fali cylindrycznej\\
  $\bf{S}_p$ & -- & wektor gêstoœci strumienia mocy skojarzony z~polem promieniowania\\
  $S$ & -- & wartoœæ œrednia gêstoœci strumienia mocy\\
  $s$ & -- & odleg³oœæ \\
  SAR & -- & tempo poch³aniania na jednostkê masy\\
  $T$ & -- & wspó³czynnik transmisji\\
  $u$ & -- & funkcja, prêdkoœæ rozchodzenia siê fali w~oœrodku\\
  $[V]$ & -- & wektor pobudzenia \\
  $w$ & -- & szerokoϾ anteny\\
  $w_n$ & -- & $n$-ta funkcja wagowa\\
  $x,~y,~z,$ & -- & wspó³rzêdne prostok¹tne punktu \\
  $[Z]$ & -- & macierz impedancyjna tworzona metod¹ momentów \\
  ${\bf 1}_{p}$ & -- & wektor jednostkowy okreœlaj¹cy polaryzacjê pola \\%wytwarzanego przez $i$-te Ÿród³o\\
  $\nabla$ & -- & operator nabla\\
  $\alpha$ & -- & t³umiennoœæ oœrodka\\
  $\alpha_n$ & -- & $n$-ty wspó³czynnik rozwiniêcia\\
  $\Delta x$ & -- & krok dyskretyzacji przestrzeni\\
  $\Delta t$ & -- & krok dyskretyzacji czasu\\
  $\epsilon$ & -- & przenikalnoϾ elektryczna \\
  $\varepsilon$ & -- & b³¹d wzglêdny
 \end{tabular}
}
\newpage
\thispagestyle{empty}%<---------------------------- pusta strona na odwrocie tutu³owej
%\pagestyle{fancy}
\cleardoublepage%<--------------------------------- dalej bêdzie od nieparzystej

%-------------------------------------------------------------------------------
% Wstawiane poszczególne rozdzia³y pracy
%
\chapter{Wst�p}

\section{Telefon kom�rkowy jako platforma}
telefony kom�rkowe powsta�y z my�l� o komunikacji bezprzewodowej.
Mo�liwo�� wykonywania zwyk�ych po��cze� telefonicznych z dowolnego
miejsca sta�a si� bardzo u�ytecznym narz�dziem w biznesie a wkr�tce
i w �yciu codziennym.

Z biegiem czasu producenci telefon�w kom�rkowych umieszczali coraz
to lepsze procesory, coraz wi�cej pami�ci, fakt ten wykorzystali
programi�ci. Dynamicznie rozwijaj�cy si� rynek gier wymusi�
zastosowanie kolorowych wy�wietlaczy LCD, joystick'�w, za�
zastosowania biznesowe wnios�y takie rozwi�zania jak mini
klawiaturki 'qwerty', rysiki, ekrany dotykowe itp. Obecnie
standardem s� telefony wyposa�one w system operacyjny jak Symbian OS
czy Windows Mobile. Telefon kom�rkowy przesta� by� tylko urz�dzeniem
do komunikacji radiowej, lecz sta� si� pot�nym narz�dziem w
kieszeni u�ytkownika.
%
\section{Programowanie mobilne}
\label{sec:Programowanie mobilne} Pisanie program�w na platformy
mobilne jest oparte na j�zykach wy�szego rz�du, jak Java, C++.
Poniewa� jednak platformy mobilne takie jak telefony kom�rkowe,
smartfony czy palmtopy maj� w por�wnaniu do komputer�w klasy PC
ograniczone mo�liwo�ci, programowanie mobilne rz�dzi si� swoimi
prawami. W przeciwie�stwie do platform stacjonarnych kt�rych
mo�liwo�ci obliczeniowe mog� by� w miar� potrzeb rozbudowywane
poprzez zastosowanie lepszych procesor�w i wi�kszej ilo�ci pami�ci,
platformy mobilne maj� zazwyczaj bardzo ograniczone zasoby. Z tego
wzgl�du zastosowano wiele ogranicze�, znikn�y niekt�re klasy znane
z programowania pecetowego, pojawi�y si� ich modyfikacje lub
zupe�nie nowe klasy, stworzone na potrzeby zastosowa� mobilnych. Kod
aplikacje mobilnych powinien by� zoptymalizowany tak aby zu�ywa�y
jak najmniej pami�ci i mocy obliczeniowej procesora przy czym musz�
one by� niezawodne i dzia�a� szybko oszcz�dzaj�c czas u�ytkownika.
Programuj�c na platformy mobilne musimy zrezygnowa� ze
skomplikowanych oblicze�, przy mo�liwo�ciach wsp�czesnych telefon�w
kom�rkowych �atwo zapomnie� o ograniczeniach i aplikacja b�dzie
dzia�a� wolno lub po prostu b�dzie si� ,,wiesza�'' pomimo poprawnego
kodu.

%
%\chapter{Cel i zakres pracy} \label{sec:Cel i zakres pracy}
Celem pracy jest stworzenie aplikacji
mobilnej w j�zyku Java 2 Micro Edition prezentuj�cej mo�liwo�ci
wykorzystania standardu Bluetooth implementowanej w telefonach
kom�rkowych. Zakres pracy obejmuje r�wnie� opis wykorzystania
platformy Java w programowaniu mobilnym Ze szczeg�lnym
uwzgl�dnieniem klas \textbf{MediaAPI} umo�liwiaj�cej nagrywanie oraz
odtwarzanie d�wi�ku oraz \textbf{BluetoothAPI} b�d�cej implementacj�
wi�kszo�ci znanych serwis�w Bluetooth.

%
%\chapter{Bluetooth}
\label{sec:Bluetooth}
%
\section{Wstêp}
\label{sec:Wstep}
%
 Technologia Bluetooth jest standardem umo¿liwiaj¹cym bezprzewodow¹ komunikacjê
pomiêdzy ró¿nymi urz¹dzeniami elektronicznymi, takimi jak
klawiatura, komputer, laptop, palmtop, telefon komórkowy czy
urz¹dzenia audio. Interfejs Bluetooth pracuje w nielicencjonowanym
paœmie ISM (Industrial Scientific Medicine). W wiêkszoœci krajów
jest to zakres 2400 - 2483,5 MHz (wyj¹tkiem jest np. Francja i
Hiszpania, gdzie pasmo to ma zakres 2446,5 - 2483,5 MHz). Okreœlone
pasmo jest podzielone na 79 (Francja, Hiszpania: 23) kana³ów o
szerokoœci 1 MHz, które s¹ wykorzystywane w technice frequency
hopping. Polega ona na okresowej zmianie kana³u transmisyjnego w
celu unikniêcia zak³óceñ. W standardzie zmiana kana³u nastêpuje 1600
razy na sekundê , a urz¹dzenia maj¹ do wyboru 5 ró¿nych sekwencji
zmiany kana³ów tzw. hopping sequence, zgodnie z którymi dokonywane
s¹ zmiany czêstotliwoœci. Frequency hopping pozwala zmniejszyæ
zak³ócenia transmisji wyp³ywaj¹ce z pracy takich urz¹dzeñ jak
kuchenki mikrofalowe czy modu³y 802.11, pracuj¹ce w tym samym
zakresie czêstotliwoœci. Urz¹dzenia Bluetooth pozwalaj¹ na pracê w
zakresie mocy od 1 do 100 mW, co umo¿liwia transmisjê na odleg³oœci
od 10 do 100 m. Wiêkszoœæ urz¹dzeñ Bluetooth ogranicza jednak pobór
mocy do minimum i pozwala na pracê w odleg³oœciach do 10 m,
zmniejsza to tak¿e koszty produkcji uk³adu, gdy¿ nie musi on byæ
wyposa¿ony w modu³ kontroli mocy.

Zasiêg Bluetooth zale¿y od klasy mocy:

\begin{itemize}
    \item klasa 1 (100 mW) ma najwiêkszy zasiêg, do 100 m,
    \item klasa 2 (2,5 mW) jest najpowszechniejsza w u¿yciu, zasiêg do 10 m
    \item klasa 3 (1 mW) rzadko u¿ywana, zasiêg do 1 m.
\end{itemize}

 Transfer maksymalny okreœla edycja protoko³u
\begin{itemize}
    \item Bluetooth 1.0 - 721 kb/s
    \item Bluetooth 1.1 - 721 kb/s
    \item Bluetooth 1.2 - 721 kb/s
    \item Bluetooth 2.0 - transfer maksymalny przesy³ania danych na poziomie 2.1~Mb/s, Enhanced Data Rate - transfer do 3.0 Mb/s
\end{itemize}

Protokó³ Bluetooth nie jest jednak doskona³y. Zawiera w sobie luki,
które umo¿liwiaj¹ dostêp  z zewn¹trz do przesy³anych danych. Pakiety
Bluetooth mo¿na stosunkowo ³atwo przechwyciæ przy pomocy
specjalnych programów tzw. snifferów. Ponadto niektóre urz¹dzenia
wyposa¿one w Bluetooth, g³ównie ze wzglêdu na prostotê obs³ugi,
umo¿liwiaj¹ nawi¹zanie z nimi po³¹czenia bez jakiejkolwiek wiedzy
ich w³aœciciela. Istnieje jednak mo¿liwoœæ u¿ytkowania protoko³u
Bluetooth w taki sposób, ¿e ewentualne przechwycenie danych nie
stanowi ¿adnego zagro¿enia dla przesy³anych danych. Wystarczy
opakowaæ wysy³ane dane u¿ywaj¹c odpowiednich technik w warstwie
aplikacji. Wówczas pods³uchane dane po stronie ,,sniffera'' to
zaledwie ciąg bezu¿ytecznych bajtów.
%
\subsection{Topologia}
\label{sec:Mozliwosci} Komunikacja w technologii Bluetooth odbywa
siê na zasadzie master-slave. Po nawi¹zaniu po³¹czenia jedno z
urz¹dzeñ (inicjator po³¹czenia) staje siê masterem, drugie slavem
(przy czym urz¹dzenia Bluetooth s¹ pod tym wzglêdem nierozró¿nialne,
ka¿de mo¿e pe³niæ obie funkcje). Standard definiuje pojêcie piconetu
tj. sieci sk³adaj¹cej siê z jednego mastera i od jednego do siedmiu
aktywnych slave'ów, przy czym komunikacja zawsze odbywa siê poprzez
mastera. Nie ma wiêc mo¿liwoœci bezpoœredniej komunikacji pomiêdzy
slave'ami w ramach piconetu. Wiele piconetów tworzy poprzez
wspó³dzielenie wêz³ów wiêksz¹ sieæ tzw. scatternet. jak pokazano na
rysunku poni¿ej wêze³ po³¹czony z wiêcej ni¿ jednym piconetem mo¿e
byæ slavem we wszystkich z nich, lub masterem w jednym piconecie i
slave'm we wszystkich pozosta³ych.

\begin{figure}[b]
\centering
    \includegraphics[width=50mm]{kot_d.jpg}
     \caption{Ka¿dy rysunek musi mieæ swój podpis!!}\label{F_cube}
\end{figure}


W zwi¹zku z tym, ¿e sieci Bluetooth tworzone s¹ ad-hoc a technologia
przeznaczona jest g³ównie dla urz¹dzeñ o charakterze przenoœnym
(mobilnym), elementy pico-sieci mog¹ czêsto ,,wychodziæ'' z zasiêgu
innych urz¹dzeñ lub siê w nim niespodziewanie ,,pojawiaæ''. Topologia
sieci i urz¹dzenia, które j¹ tworz¹, ulegaj¹ wiêc ci¹g³ym zmianom.
Protokó³ Bluetooth musi zawieraæ zatem kontroler  który ³¹czy
wykorzystuj¹c informacje wysy³ane przez ka¿de urz¹dzenie Bluetooth w
zasiêgu, zarz¹dza procesami wykrywania i kojarzenia urz¹dzeñ.
%
\section{Stos protoko³ów}\label{sec:stos}
Protokó³ Bluetooth jest wykorzystywany w urz¹dzeniach które mog¹ siê
ró¿nic technologicznie w zale¿noœci od zastosowania lub producenta.
Stos protoko³ów Bluetooth nie odpowiada ¿adnemu znanemu modelowi
(OSI, TCP/IP, 802) ale w przeciwieñstwie do wiêkszoœci protoko³ów
sieciowych, które okreœlaj¹ tylko kana³y pomiêdzy komunikuj¹cymi
siê jednostkami, stos protoko³ów Bluetooth w wersji 1.1 okreœla 13
specjalnych aplikacji, zwanych profilami systemu Bluetooth, w
których system mo¿e byæ u¿ywany.
%
\subsection{Warstwa Fizyczna}
\label{sec:Warstwa Fizyczna}

Warstwê fizyczn¹ Bluetooth stanowi interfejs radiowy o stosunkowo
niewielkim poborze mocy, dzia³aj¹cy na w zale¿noœci od klasy w
zasiêgu 1 - 100m. Warstwa radiowa Bluetooth operuje w paœmie ISM
2,4 GHz. Pasmo to jest podzielone na 79 kana³ów, po 1MHz ka¿dy.
System wykorzystuje modulacje FSK (Frequency Shift Keying) dziêki
której uzyskujemy prêdkoœci transmisji do 1 Mbit/s. Skakanie po
czêstotliwoœci odbywa siê z czêstoœci¹ 1600 skoków na sekundê.
Sekwencjê skoków dyktowane s¹ przez wêze³ master.
%
\subsection{Warstwa Baseband}
\label{sec:Warstwa Baseband} Warstwa baseband jest zbli¿ona do
podwarstwy MAC modelu OSI. Upakowuje ona luŸne bity w ramki. Master
w ka¿dej pikosieci definiuje sloty czasowe o d³ugoœci $625 \mu s$.
Transmisja mastera zaczyna siê od slotów parzystych natomiast
transmisja slave od slotów nieparzystych (multipleksacja w
dziedzinie czasu dziedzinie czasu - TDM). Ramki mog¹ mieæ d³ugoœæ
jednego, trzech lub piêciu slotów czasowych. Skakanie czêstotliwoœci
pozwala ustawiæ czas $250 - 260 \mu s$ na skok, aby umo¿liwiæ
stabilizacjê uk³adów radiowych. Ka¿da ramka jest transmitowana
przez kana³ logiczny, nazywany z angielskiego ,,link'', pomiêdzy
masterem i urz¹dzeniem slave. Istniej¹ dwa rodzaje kana³ów
logicznych. Pierwszy nazywa siê ACL (Asynchronous Connection-Less),
u¿ywany w po³¹czeniu z komutacj¹ pakietów, gdzie dane s¹ dostêpne w
nieregularnych odstêpach czasu. Dane te pochodz¹ od warstwy L2CAP po
stronie nadawczej i s¹ dostarczane do warstwy L2CAP po stronie
odbiorczej. W tej wersji kana³u logicznego nie ma ¿adnych gwarancji,
ze ramka dotrze do celu. Ramki mog¹ zostaæ utracone i wymagaæ
retransmisji. Urz¹dzenie slave mo¿e mieæ tylko jeden kana³ typu ACL
z urz¹dzeniem master. Drugi typ kana³u logicznego to SCO
(Synchronous Connection Oriented) i jest u¿ywany do transmisji w
czasie rzeczywistym, np. rozmowy telefonicznej z u¿yciem zestawu
s³uchawkowego. Ramki transmitowane w tego typu kanale, nie mog¹ byæ
retransmitowane. Zamiast tego mo¿na stosowaæ korektê b³êdów, aby
zapewniæ mo¿liwie wysok¹ niezawodnoœæ. Urz¹dzenie slave mo¿e
korzystaæ z maksymalnie trzech kana³ów typu SCO w kierunku mastera.
Ka¿de  SCO mo¿e transmitowaæ jeden kana³ telefoniczny (PCM,
64kbit/s).
%
\subsection{Protokó³ Voice}
\label{sec:Protokol Voice}

Protokó³ Bluetooth umo¿liwia przesy³anie sygna³ów wymagaj¹cych
transmisji synchronicznej, takich jak sygna³ mowy, a tak¿e sygna³ów,
które wymagaj¹ wiêkszych prêdkoœci transmisji, ale mog¹ byæ
przesy³ane asynchronicznie, czyli transmisji danych. Zdefiniowane s¹
dwa typy ³¹cz:
\begin{itemize}
\item  ³¹cza synchroniczne (SCO, Synchronous Connection Oriented),
\item  ³¹cza asynchroniczne (ACL, Asynchronous Connectionless).
\end{itemize}
Dostêpne s¹ pakiety ACL dwóch typów: DM (Data Medium) i DH (Data
High). Pakiety DH zawieraj¹ mniej informacji zwi¹zanych z korekcj¹
b³êdów, dziêki czemu umo¿liwiaj¹ wiêksz¹ prêdkoœæ transmisji.
Najszybsza transmisje umo¿liwiaj¹ pakiety DH5, które zajmuj¹ piêæ
szczelin czasowych. Pakiet taki mo¿e przenieœæ 339 bajtów (2712
bitów) danych, a wiec, aby przes³aæ 2712 bitów informacji, trzeba
przes³aæ 2854 bitów). W odpowiedzi na pakiet DH5, który zajmuje 5
szczelin, musi zostaæ przes³any pakiet o d³ugoœci co najmniej jednej
szczeliny. Daje to maksymalna przepustowoœæ w jednym kierunku równa
723,2 kb/s. W kierunku zwrotnym pakiety o d³ugoœci jednej szczeliny
przenios¹ 57,6 kb/s. Jeœli pakiety DH5 bêd¹ przesy³ane w obu
kierunkach, uzyskana przepustowoœæ w jednym kierunku bêdzie wynosi³a
433,9 kb/s. Ze wzglêdu na to, ze wy¿sze warstwy protoko³u Bluetooth
wymagaj¹ przesy³ania danych steruj¹cych, maksymalna przepustowoœæ na
poziomie aplikacji wynosi oko³o 650 kb/s. Koszty ponoszone na
kodowanie danych i skoki czêstotliwoœci s¹ niezbêdne, aby utworzone
³¹cze by³o niezawodne, poniewa¿ wykorzystywane pasmo ISM jest
wspó³u¿ytkowane przez wiele ró¿nego rodzaju urz¹dzeñ. Ograniczenia
s¹ tak¿e wymuszane przez lokalne przepisy, które ograniczaj¹ moc
promieniowana w jednostce czasu w pasmie ISM. Dziêki skokom
czêstotliwoœci transmisja jest rozpraszana w czêstotliwoœci i w
czasie. £¹cza SCO dzia³aj¹ z przepustowoœci¹ 64 kb/s. W jednym
czasie mo¿na korzystaæ z trzech ³¹czy g³osowych. Mo¿na tak¿e mieszaæ
w jednym ³¹czu dane g³osowe przesy³ane synchronicznie i dane
przesy³ane asynchronicznie. £¹cza SCO stanowi¹ symetryczne ³¹cza
miêdzy urz¹dzeniem nadrzêdnym i podrzêdnym o zarezerwowanej
przepustowoœci i zapewniaj¹ regularn¹, w okreœlonych odstêpach
czasu, wymianê danych w zarezerwowanych szczelinach czasowych. S¹ to
wiec ³¹cza umo¿liwiaj¹ce przekazywanie danych, dla których istotny
jest czas, a wiec na przyk³ad sygna³ów audio. Urz¹dzenie nadrzêdne
mo¿e obs³ugiwaæ do trzech ³¹czy z tym samym lub ró¿nymi urz¹dzeniami
podrzêdnymi, a urz¹dzenie podrzêdne mo¿e obs³ugiwaæ do trzech ³¹czy
z urz¹dzeniem nadrzêdnym. Pakiety SCO nie s¹ nigdy retransmitowane.
Protokó³ ten nie jest umieszczony w zestawie klas J2ME i nie mo¿e
byæ wykorzystany w aplikacjach mobilnych.

%
\subsection{HCI}
\label{sec:Warstwa HCI}

\textbf{Host Controller Transport} pozwala na transparentn¹ wymianê
informacji takich jak komendy HCI, dane ACL i SCO, dostarczenie
komend kontroluj¹cych pracê warstwy baseband i Link Manager. G³ównym
zadaniem HCI jest dostarczenie metod umo¿liwiaj¹cych korzystanie z
zasobów warstwy baseband protoko³u Bluetooth.

\subsection{LMP}
\label{sec:Warstwa LMP} \textbf{Link Manager Protocol} ustala
parametry ³¹cza, umo¿liwia nas³uchiwanie i wykrywanie urz¹dzeñ w
pobli¿u. LMP sk³ada siê z kilku PDU (protocol Data Units) które s¹ wysy³ane z
jednego urz¹dzenia do innego znajduj¹cego siê w zasiêgu w procesie
nawi¹zywania po³¹czenia. PDU s¹ wysy³ane jako single-slot z
jednobajtowym nag³ówkiem.
%
\subsection{Warstwa L2CAP}
\label{sec:Warstwa L2CAP}

Logical Link Control and Adaptation Layer Protocol jest warstw¹
maj¹c¹ za zadanie transmisjê danych do warstw wy¿szych zarówno w
sposób zorientowany po³¹czeniowo jak np. port szeregowy RFCOMM, jak
równie¿ bez po³¹czeniowo. Warstwa ta zawiera równie¿ zestaw metod do
multipleksowania i segmentacji danych oraz operacji odwrotnych
potrzebnych do przesy³ania danych pomiêdzy warstw¹ baseband a
warstwami wy¿szymi.

 Warstwa L2CAP spe³nia trzy g³ówne funkcje:
\begin{itemize}
  \item przyjmuje pakiety o maksymalnym rozmiarze do 64 KB od wy¿szych warstw i dzieli je na ramki w celu transmisji.
   Na koñcu ramki s¹ ponownie sk³adane w ca³oœæ.
  \item zajmuje siê multipleksacj¹ i demultipleksacj¹ z³o¿onych pakietów. Gdy pakiet jest sk³adany w ca³oœæ, warstwa L2CAP okreœla, któremu protoko³owi warstwy wy¿szej go przekazaæ, np. do RFcomm lub telephony.
  \item zajmuje siê wymaganiami na jakoœæ us³ugi, zarówno podczas zestawiania po³¹czenia oraz podczas realizacji us³ugi.
\end{itemize}

\emph{Segmentacja}

W porównaniu do innych mediów fizycznych, pakiety danych
zdefiniowane przez warstwê baseband maj¹ ograniczony rozmiar.
Wysy³anie maksymalnej iloœci danych zdefiniowane jest przez MTU
(maximum transmission unit) i przy maksymalnym wykorzystaniu pasma
wynosi 341 bajtów na pakiet DH5. Wy¿sze warstwy z regu³y
przystosowane s¹ do obs³ugi znacznie wiêkszych "porcji" danych,
dlatego du¿e pakiety L2CAP musz¹ byæ podzielone na segmenty tak aby
ich przesy³ drog¹ radiow¹ by³ mo¿liwy przy okreœlonych parametrach
³¹cza. W podobny sposób segmenty odbierane z warstwy baseband musz¹
byæ ponownie posk³adane w jeden du¿y pakiet warstwy L2CAP. Dane
które z regu³y przesy³amy za pomoc¹ Bluetooth wykorzystuj¹ protoko³y
których pakiety s¹ znacznie wiêksze ni¿ wielkoœæ segmentów w
warstwie baseband. Dlatego segmentacja i sk³adanie pakietów SAR
(Segmentation and Reassembly) jest absolutnie niezbêdne.

\emph{Quality of Service}

Protokó³ L2CAP podczas nawi¹zywania po³¹czenia negocjuje dostêpne
serwisy: QoS (Quality of Service). Ka¿da implementacja warstwy L2CAP
musi udostêpniaæ informacje na temat serwisów które mog¹ zostaæ
wykorzystane podczas transmisji i byæ w stanie wys³aæ potwierdzenie
do urz¹dzenia wysy³aj¹cego ¿¹danie. Dziêki temu modu³y bluetooth
mog¹ siê ze sob¹ komunikowaæ niezale¿nie od implementacji i
konfiguracji sprzêtowej.


 \emph{Identyfikator kana³u}

Identyfikatory kana³ów CIDs (Channel IDentifiers) to nazwy lokalne
reprezentuj¹ce kana³ logiczny punktu koñcowego danego urz¹dzenia.
Implementacja L2CAP mo¿e dowolnie zarz¹dzaæ CID pod warunkiem ¿e
nazwa ta pozostanie unikalna, tzn nie zostanie u¿yta jako punkt
koñcowy w innym po³¹czeniu w przypadku wielokana³owego po³¹czenia
L2CAP.

Przypisywanie CID jest zale¿ne od danego urz¹dzenia i mo¿e byæ
przydzielone niezale¿nie od innych urz¹dzeñ (z wyj¹tkiem kana³ów
zarezerwowanych jak np. kana³ sygnalizacyjny). jednak¿e nawet jeœli
w zasiêgu znajd¹ siê wiele urz¹dzeñ o takiej samej wartoœci CID,
ka¿de z tych urz¹dzeñ mo¿e siê bez trudu po³¹czyæ z dowolnym innym o
unikalnej nazwie CID.

Kana³y przesy³u danych zorientowane po³¹czeniowo, reprezentuj¹
po³¹czenie dwóch urz¹dzeñ fizycznych, gdzie CID s¹ identyfikatorami
ka¿dego z wêz³ów koñcowych danego kana³u. kana³y bezpo³¹czeniowe
pozwalaj¹ na transmisjê danych tylko w jednym kierunku i s¹ u¿ywane
do po³¹czeñ grupowych. Wówczas CID po stronie Ÿród³a reprezentuje
jedno lub wiele urz¹dzeñ do których wysy³ane s¹ dane. Istnieje tak¿e
kilka kana³ów specjalnych jak np kana³ sygnalizacyjny który
umo¿liwia negocjacjê w fazie nawi¹zywania po³¹czenia dlatego ka¿da
implementacja L2CAP musi koniecznie obs³ugiwaæ ten kana³. Pozosta³e
kana³y specjalne to kana³y zarezerwowane na wszystkie mo¿liwe dane
przychodz¹ce, przesy³ane w sposób bezpo³¹czeniowy.

 \emph{Sygnalizacja}

Warstwy L2CAP dwóch urz¹dzeñ wysy³aj¹ komendy sygnalizacyjne
u¿ywaj¹c kana³u o CID 0x0001 (kana³ sygnalizacyjny). Ka¿da
implementacja L2CAP musi umieæ wyodrêbniæ adres Bluetooth (BD ADDR)
urz¹dzenia nadaj¹cego komendê sygnalizacyjn¹. pakiety mog¹ sk³adaæ
siê z kilku komend i byæ wys³ane w ten sam sposób na ten sam adres
kana³owy CID. Jako komendy sygna³owe wysy³ane s¹ tak¿e komendy MTU
które wymagaj¹ potwierdzenia ze strony odbiorcy. Warstwa L2CAP
dostarcza nastêpuj¹ce serwisy

\begin{itemize}
  \item Po³¹czenie: \emph{setup , configure , disconnect}
  \item Dane: \emph{read  , write}
  \item Grupy: \emph{create, close, add member, remove member , get membership}
  \item Informacje: \emph{ping, get info, request a call-back at the occurrence of an event}
  \item Transmisja bezpo³¹czeniowa: \emph{enable, disable}
\end{itemize}

%
\subsection{£¹cze szeregowe RFCOMM}
\label{sec:Lacze szeregowe RFCOMM}
RFCOMM (Radio Frequency Communication) jest prostym protoko³em transportowym dostarczaj¹cym emulacjê portu szeregowego RS232 przy wspó³pracy z protoko³em L2CAP. Protokó³ ten jest oparty o standard ETSI TS 07.10. protokó³ RFCOMM
umo¿liwia nawi¹zanie do 60 po³¹czeñ jednoczeœnie.

Urz¹dzenia bluetooth u¿ywaj¹ protoko³u RFCOMM jako emulacjê
wielokrotnego portu szeregowego. protokó³ ten umo¿liwia otwarcie do
60 emulowanych portów jednoczeœnie. Do kontroli po³¹czeñ pomiêdzy
aplikacjami serwera i klienta s³u¿y Data Link Connection Identifier
(DLCI) DLCI to 6 bitowy numer reprezentuj¹cy wartoœci od 1 do 61 i
stanowi unikalny adres ka¿dej otwartej sesji RFCOMM pomiêdzy dwoma
urz¹dzeniami.

 Analogicznie do portu szeregowego RS232 gdzie funkcje
steruj¹ce przep³ywem s¹ realizowane przy pomocy ramek RTS/CTS, port
RFCOMM równie¿ posiada kontrolê przep³ywu która jest realizowana
przez warstwê L2CAP. Kontrola przep³ywu danych mo¿e siê ró¿niæ w
zale¿noœci od implementacji warstwy L2CAP. RFCOMM u¿ywa protoko³ów w
warstwie L2CAP do skojarzenia kana³ów tej warstwy potrzebnych do
nawi¹zania po³¹czenia z portem RFCOMM zdalnego urz¹dzenia. Po
otwarciu odpowiedniego kana³u warstwy L2CAP na potrzeby komunikacji
szeregowej RFCOMM rozpoczyna komunikacjê na zasadzie sesji
multipleksowanej RFCOMM/TS 07.10.

Poniewa¿ komunikacja RFCOMM w ca³oœci opiera siê o serwisy
dostarczane przez warstwê L2CAP, protoko³y w tej warstwie musz¹
dzia³aæ niezawodnie aby wszystkie pakiety RFCOMM dotar³y we
w³aœciwej kolejnoœci i bez powtórzeñ. Jeœli w kanale L2CAP nast¹pi
b³¹d, warstwa RFCOMM odbiera informacjê o utracie ³¹cza.

Protokó³\textbf{ RFCOMM}oraz serwisy warstwy\textbf{L2CAP} s¹ bardzo
dobrze zaimplementowane w platformie J2ME. Ich funkcje i zalety
zosta³y szerzej opisane w rozdziale \ref{sec:Java Micro Edition}.

\section{Adres i transmisja Bluetooth}\label{sec:Adres Bluetooth}
%Ka¿de urz¹dzenie posiada 48 bitowy adres
IEEE MAC (Bluetooth Device Address, BD-ADDR). Jest on u¿ywany do
inicjowania pewnych operacji oraz obliczania kodu dostêpu. Adres MAC
jest podzielony na trzy czêœci:
\begin{itemize}
\item  Non-significant Address Part (NAP)
BD-ADDR [47:32]. NAP [15:0] która jest u¿ywana do inicjowania
szyfrowania.
\item  Upper Address Part (UAP)
BD-ADDR [31:24]. UAP [7:0] która jest u¿ywana do inicjowania
obliczeñ HEC (Header Error Check) i CRC oraz skoków czêstotliwoœci.
\item Lower Address Part
BD-ADDR [23:0]. LAP [23:0] która jest u¿ywana do generowania s³owa
synchronizuj¹cego i skoków czêstotliwoœci.
\end{itemize}

System Bluetooth jest systemem typu TDM (Time Division Multiplexed),
czyli takim, który dzia³a w oparciu o multipleksowanie w dziedzinie
czasu. Podstawow¹ jednostk¹ jest szczelina czasowa o d³ugoœci $625
\mu s$. Podczas przesy³ania danych wszystkie operacje s¹ realizowane
w ci¹gu 1, 3, 5 szczelin czasowych, które tworz¹ pakiet. W przypadku
operacji poprzedzaj¹cych transmisjê (zapytania, wywo³ania,
skanowanie) mog¹ byæ u¿ywane szczeliny czasowe o d³ugoœci po³owy
normalnych szczelin. Pakiety tworz¹ pary nadawczo-odbiorcze. Podczas
po³¹czenia taka para mo¿e siê sk³adaæ z 2, 4, 6, 8, 10 szczelin.
Ka¿de urz¹dzenie Bluetooth mo¿e w danej chwili w jednej pikosieci
pe³niæ rolê urz¹dzenia nadrzêdnego (Master) albo podrzêdnego
(Slave). Nie mo¿e jednak pe³niæ obu tych ról jednoczeœnie.
Urz¹dzenie nadrzêdne (Master) inicjuje przesy³anie danych, a
urz¹dzenie podrzêdne (Slave) odpowiada na sygna³y urz¹dzenia Master.
Urz¹dzenie podrzêdne musi byæ zsynchronizowane z urz¹dzeniem
nadrzêdnym w czasie i czêstotliwoœci. Transmisja przebiega w sposób
nastêpuj¹cy: urz¹dzenie M transmituje do urz¹dzenia S na kanale
K(n), $625 \mu s$ póŸniej oba urz¹dzenia przestrajaj¹ radia na
czêstotliwoœæ K(n+1), a nastêpnie urz¹dzenie S musi odpowiedzieæ na
poprzedni pakiet urz¹dzenia M, urz¹dzenie S nas³uchuje pakietu od M,
który mo¿e zostaæ wys³any lub nie, ale nie bêdzie transmitowaæ,
dopóki pakiet taki nie zostanie odebrany. Ka¿de urz¹dzenie realizuje
skok czêstotliwoœci raz na pakiet. Zapewnia to:
\begin{itemize}
\item Bezpieczeñstwo - poniewa¿ skoki s¹ zdefiniowane na podstawie pseudoprzypadkowej sekwencji obliczonej na podstawie adresu urz¹dzenia M,
\item Niezawodnoœæ - pakiet utracony w kanale K(n) zostanie najprawdopodobniej prawid³owo przes³any w kanale
K(n+k), poniewa¿ oba kana³y s¹ oddalone od siebie o znacz¹c¹
odleg³oœæ.
\end{itemize}
W specyfikacji Bluetooth zdefiniowane pakiety maj¹ d³ugoœci 1, 3, 5
szczelin. U¿ycie d³u¿szego pakietu zapewnia szybsza transmisje, ale
mniejsza niezawodnoœæ. Zak³ócenie powoduj¹ce wykluczenie pakietu
wyst¹pi raczej w przypadku pojedynczego d³ugiego pakietu w jednym
kanale ni¿ w kilku kolejnych kana³ach wystêpuj¹cych w sekwencji
skoków czêstotliwoœci. Wszystkie pakiety zawieraj¹ te sam¹ iloœæ
danych steruj¹cych i w nag³ówku, wiêc u¿ycie d³u¿szego pakietu jest
bardziej efektywne pod wzglêdem liczby przesy³anych danych.

\section{Nawi¹zywanie po³¹czenia}
\label{sec:Nawiazywanie polaczenia}

Urz¹dzenie które rozpoczyna proces wykrywania, wysy³a pakiety ID
zawieraj¹ce kod rozpoznawczy kod IAC (Inquiry Access Code). Zwykle
u¿ywany jest kod GIAC (General IAC), którego format jest zrozumia³y
dla wszystkich urz¹dzeñ. Z powodu ograniczonych zasobów
energetycznych jakim charakteryzuj¹ siê urz¹dzenia przenoœne,
wykrywanie nie jest realizowane periodycznie jak w wiêkszoœci sieci
z wykrywaniem stanu ³¹cza, lecz tylko wtedy, gdy zostanie ono
zainicjowane przez wy¿sze warstwy protoko³u. Stan oczekiwania na
wykrycie (inquiry scan) jest wykorzystywany okresowo w stosunkowo
krótkim oknie czasowym.

Gdy urz¹dzenie oczekuj¹ce na wykrycie (nas³uchuj¹ce) odbierze sygna³
wykrywaj¹cy, mo¿e odpowiedzieæ natychmiast. Takie dzia³anie mog³oby
jednak spowodowaæ, ¿e kilka urz¹dzeñ odpowie równoczeœnie, co
mog³oby spowodowaæ kolizjê i uniemo¿liwiæ odebranie odpowiedzi. Aby
tego unikn¹æ, stosowane jest losowe opóŸnienie. Gdy urz¹dzenie
odbierze sygna³ wykrywaj¹cy, losuje kolejne opóŸnienie,
 a nastêpnie ponownie wchodzi w stan oczekiwania na
wykrycie, i gdy odbierze nastêpny pakiet ID, odpowiada pakietem FHS.
Dziêki temu odpowiedzi ro¿nych urz¹dzeñ s¹ losowo rozrzucone w
czasie i nie zak³ócaj¹ siê wzajemnie. Niestety wi¹¿e siê to z
wyd³u¿eniem czasu pracy urz¹dzenia wykrywaj¹cego.

Urz¹dzenie wykrywaj¹ce transmituje dwa pakiety ID w jednej
szczelinie czasowej, próbuj¹c ,,trafiæ'' na urz¹dzenie oczekuj¹ce na
wykrycie. Urz¹dzenie które oczekuje na wykrycie, nas³uchuje
wykonuj¹c skoki co 2048 szczelin (co 1,28 s). Urz¹dzenie wykrywaj¹ce
transmituje pakiety ID, u¿ywaj¹c wykrywaj¹cej sekwencji skoków, i
nas³uchuje odpowiedzi pasuj¹cej do tej sekwencji skoków. Po
odebraniu pakietu ID urz¹dzenie oczekuj¹ce na wykrycie czeka przez
losow¹ liczbê szczelin czasowych, a nastêpnie ponownie przechodzi w
stan oczekiwana na wykrycie. Tym razem po odebraniu pakietu ID
odpowiada po up³yniêciu $625 \mu s$ pakietem FHS (Frequency Hopping
Sequence). Poniewa¿ urz¹dzenie wykrywaj¹ce nas³uchuje na kanale
odpowiedzi, który jest œciœle zwi¹zany z poprzednim kana³em
transmisji pakietu ID, mo¿e w ka¿dej chwili mo¿e odebraæ
potwierdzenie od urz¹dzenia wykrywanego i wys³aæ do niego pakiet
FHS. W najgorszym wypadku czas wykrywania urz¹dzenia mo¿e trwaæ 10 s
dla pojedynczego ³¹cza. Gdy urz¹dzenie obs³uguje ju¿ inne ³¹cza, na
przyk³ad SCO, czas ten mo¿e byæ odpowiednio wiêkszy.

 Podobnie jak w przypadku urz¹dzeñ znajduj¹cych siê w sieci
AppleTalk, sprzêt Bluetooth ci¹gle nas³uchuje zapytañ ze strony
innych urz¹dzeñ. Po otrzymaniu zapytania wysy³ana jest odpowiedŸ.
Urz¹dzenie wykrywaj¹ce otrzymuje status nadrzêdny (master),
natomiast wykryte ma status podrzêdny (slave). Te role s¹ niezbêdne
do synchronizacji obu urz¹dzeñ. Po nawi¹zaniu po³¹czenia wymieniane
s¹ unikalne identyfikatory oraz nazwy (je¿eli s¹ dostêpne). Zwykle
master zapytuje nastêpnie urz¹dzenie podrzêdne o listê us³ug (tzw
service discovery). Odkrywanie dostêpnych serwisów jest zale¿ne od
implementacji i mo¿liwoœci danego urz¹dzenia. W przypadku telefonu
komórkowego mo¿e to byæ mo¿liwoœæ nawi¹zywania po³¹czeñ g³osowych i
przesy³ania okreœlonych typów danych. proces ten znalaz³ obszern¹
implementacjê w J2ME pod nazw¹ Service Discovery która zosta³a
opisana szerzej w rozdziale Java 2 Micro Edition.
%
%

Aby mo¿na by³o nawi¹zaæ po³¹czenie, urz¹dzenie inicjuj¹ce musi
skierowaæ bezpoœrednie ¿¹danie do innego urz¹dzenia. W tym samym
czasie wywo³ywane urz¹dzenie musi nas³uchiwaæ wywo³añ (page scan).
Po zakoñczeniu procesu wywo³ywania urz¹dzenie wywo³uj¹ce staje siê
urz¹dzeniem nadrzêdnym a urz¹dzenie wywo³ywane urz¹dzeniem
podrzêdnym. Mo¿e to ulec zmianie w wyniku prze³¹czenia ról urz¹dzeñ.
Polecenie utworzenia po³¹czenia powoduje przejœcie urz¹dzenia w tryb
wywo³ywania (paging), w którym wysy³a ono serie pakietów
wywo³uj¹cych (pakietów ID opartych na adresie urz¹dzenia
wywo³ywanego). Urz¹dzenie wywo³ywane musi byæ skonfigurowane do
oczekiwania na wywo³anie o okreœlonym czasie trwania i okreœlonych
odstêpach. Gdy urz¹dzenie wywo³ywane odbierze pakiet ID ze swoim
adresem, odpowiada pakietem ID równie¿ ze swoim adresem. Poniewa¿
adresy te s¹ unikalne dla ka¿dego urz¹dzenia, nie ma
niebezpieczeñstwa, ¿e na wywo³anie odpowie kilka urz¹dzeñ na raz.
Urz¹dzenie wywo³uj¹ce po odebraniu potwierdzenia w postaci pakietu
ID wysy³a pakiet FHS. Urz¹dzenie wywo³ywane potwierdza odebranie
tego pakietu ponownie pakietem ID po czym jest w stanie, na
podstawie informacji zawartych w pakiecie FHS, obliczyæ sekwencjê
skoków czêstotliwoœciowych urz¹dzenia wywo³uj¹cego. Urz¹dzenie mo¿e
tak¿e przestaæ korzystaæ z sekwencji wywo³ywania i przejœæ na now¹
sekwencjê zwi¹zan¹ z urz¹dzeniem nadrzêdnym. Po nawi¹zaniu
po³¹czenia urz¹dzenie wywo³uj¹ce staje siê urz¹dzeniem nadrzêdnym, a
urz¹dzenie wywo³ywanie urz¹dzeniem podrzêdnym. Po przejœciu do nowej
sekwencji skoków urz¹dzenie nadrzêdne wysy³a pakiet POLL, aby
sprawdziæ, ¿e prze³¹czenie na now¹ sekwencjê skoków czêstotliwoœci
zosta³o wykonane poprawnie. Urz¹dzenie podrzêdne musi wtedy
odpowiedzieæ dowolnym pakietem ACL. Zwykle jest to pakiet NULL.

\section{Stany ³¹czy}
 \emph{Standby} W tym stanie urz¹dzenie jest
nieaktywne, ¿adne dane nie s¹ przesy³ane i radio jest wy³¹czone.
Urz¹dzenie nie wykrywa transmitowanych pakietów. Stan ten jest
g³ównie u¿ywany podczas pracy w trybie niskiego zasilania.

\emph{Inquiry} (wykrywanie dostêpnych urz¹dzeñ) Inquiry jest to
proces, w którym urz¹dzenie wykrywa wszystkie urz¹dzenia Bluetooth
znajduj¹ce siê w zasiêgu. W tym czasie przy u¿yciu protoko³u SDP
tworzona jest lista urz¹dzeñ, z którymi mo¿na nawi¹zaæ po³¹czenie.
Podczas tej procedury wykrywane urz¹dzenia przekazuj¹ urz¹dzeniu
wykrywaj¹cemu pakiety FHS. Umo¿liwia to utworzenie tablicy
zawieraj¹cej informacje niezbêdne do nawi¹zania po³¹czenia z danym
urz¹dzeniem, czyli wartoœæ CLKN i adres \textbf{BD ADDR}. Inquiry
Scan (oczekiwanie na wykrycie).

\emph{Stan Inquiry Scan} stanowi druga stronê procedury wykrywania.
Urz¹dzenia okresowo przechodz¹ w ten stan, aby umo¿liwiæ dostêp do
siebie urz¹dzeniom, które s¹ w stanie Inquiry. Nas³uchuj¹ one przez
pewien d³u¿szy czas (poniewa¿ nie maj¹ informacji umo¿liwiaj¹cych
synchronizacjê w czasie i czêstotliwoœci) na nadejœcie pakietów z
kodem GIAC lub DIAC. Po odebraniu poprawnej wiadomoœci przechodz¹ w
stan \emph{Inquiry Response} (odpowiedzi na wykrycie) i wysy³aj¹
pakiet FHS. W stanach tych jest wykorzystywana specjalna sekwencja
skoków czêstotliwoœciowych (szybka dla urz¹dzeñ wykrywaj¹cych, a
wolna dla urz¹dzeñ wykrywanych), która zosta³a tak zaprojektowana
aby skróciæ czas potrzebny na dopasowanie czêstotliwoœci.

\emph{Page} (wywo³ywanie) Aby nawi¹zaæ po³¹czenie, urz¹dzenie, które
ma byæ urz¹dzeniem nadrzêdnym, musi wykonaæ procedurê wywo³ywania. W
pierwszej kolejnoœci przechodzi ono w stan wywo³ywania, w którym
wysy³a komunikaty wywo³uj¹ce skierowane do okreœlonego urz¹dzenia
podrzêdnego. Wykorzystywane s¹ informacje zebrane podczas procedury
wykrywania. Rz¹dzenie podrzêdne potwierdza komunikaty wywo³uj¹ce i
urz¹dzenie nadrzêdne przechodzi w stan Master response (odpowiedzi
urz¹dzenia nadrzêdnego) i odpowiada przy u¿yciu pakietu FHS.

\emph{PageScan} (oczekiwanie na wywo³anie) Podobnie jak w stanie
oczekiwania na wykrycie urz¹dzenie przechodzi w ten stan okresowo,
aby umo¿liwiæ innym urz¹dzeniom nawi¹zanie po³¹czenia. Po pomyœlnym
odebraniu pakietu wywo³uj¹cego, urz¹dzenie przechodzi w stan Slave
response (odpowiedzi urz¹dzenia podrzêdnego), w którym potwierdza
odebranie pakietu i oczekuje pakietu FHS. Po odebraniu pakietu FHS
synchronizuje swój zegar CLK i ustawia kod dostêpu, aby przejœæ w
stan po³¹czenia. W procedurze wywo³ywania wykorzystywana jest
specjalna sekwencja skoków czêstotliwoœciowych. Procedurê
wywo³ywania mo¿na przeprowadziæ bez uprzedniego wykrywania, gdy
adres urz¹dzenia jest znany. Mo¿e tak byæ w przypadku dwóch
urz¹dzeñ, które s¹ specjalnie zaprojektowane do wspólnej pracy.
Procedura taka mo¿e jednak trwaæ trochê d³u¿ej (w teorii 10 s.)
poniewa¿ urz¹dzenie nadrzêdne nie bêdzie dysponowa³o estymata zegara
urz¹dzenia podrzêdnego. Po³¹czenie – Active (aktywne) Przy przejœciu
w stan po³¹czenia urz¹dzenie podrzêdne prze³¹cza siê na zegar CLK
urz¹dzenia nadrzêdnego, dodaj¹c offset do w³asnego zegara CLKN i tym
samym zaczyna u¿ywaæ sekwencji skoków czêstotliwoœciowych urz¹dzenia
nadrzêdnego. Rz¹dzenie nadrzêdne przesy³a pakiet POLL, aby
sprawdziæ poprawnoœæ utworzonego ³¹cza. Urz¹dzenie podrzêdne musi
wtedy odpowiedzieæ, zwykle pakietem NULL. Jeœli nie, nawi¹zanie
po³¹czenia nie powiedzie siê, po up³yniêciu okreœlonych limitów
czasu nast¹pi powrót do stanu wywo³ywania i ca³y proces wywo³ywania
zostanie powtórzony.

\emph{Po³¹czenie–Hold}(wstrzymane) W trybie Hold urz¹dzenie
przestaje obs³ugiwaæ transmisje ACL przez zdefiniowany okres, aby
zwolniæ pasmo dla innych operacji, takich jak wykrywanie,
wywo³ywanie itd. Po up³yniêciu okresu wstrzymania dzia³ania
urz¹dzenie synchronizuje siê z kodem CAC i ponownie rozpoczyna
pracê.

\emph{Po³¹czenie – Sniff} (okresowe nas³uchiwanie) W tym trybie
urz¹dzeniu podrzêdnemu jest przypisywane pewne szczeliny czasowe, w
których nas³uchuje transmisji. Urz¹dzenie nas³uchuje od szczeliny o
numerze Dsniff co Tsniff szczelin przez pewien czas równy Nsniff.
Jeœli w wyznaczonych szczelinach odebrany zostanie pakiet kierowany
do tego urz¹dzenia, odbiera ono wszystkie pakiety z w³asnym adresem
AM, a nastêpnie czeka do nastêpnego okresu nas³uchiwania.

\emph{Po ³¹cznie – Park} (uœpienie) W tym trybie urz¹dzenie oddaje
swój adres AM i nas³uchuje transmisji tylko od czasu do czasu. Mo¿e
przejœæ w tryb niskiego zasilania. Musi siê tylko uaktywniæ co
pewien czas, aby zsynchronizowaæ siê z kodem CAC, a nastêpnie
powróciæ do trybu niskiego zasilania. Dziêki temu mo¿e byæ
zsynchronizowane z urz¹dzeniem nadrzêdnym, nawet jeœli korzysta z
mniej dok³adnego zegara LPO (Low Power Oscillator).

%
%\chapter{Java 2 Micro Edition}
\label{sec:Java Micro Edition}
%
\section{Wst�p}
Na pocz�tku lat dziewi��dziesi�tych w firmie Sun Microsystems
powsta� nowy j�zyk programowania nazwany Oak. Stanowi� on cze��
projektu badawczego zajmuj�cego si� urz�dzeniami elektroniki
u�ytkowej intensywnie wykorzystuj�cej oprogramowanie. Pierwszym prototypem wykorzystuj�cym kod Oak, by� kontroler Star7. by�o to
niewielkie urz�dzenie przeno�ne wyposa�one w ekran dotykowy LCD,
wbudowany interfejs sieci bezprzewodowej oraz ��cze na podczerwie�.
Urz�dzenie to spe�nia�o funkcje elektronicznego notesu, organizera,
mog�o te� by� wykorzystywane jako pilot do telewizora, magnetowidu
itp. oprogramowanie tego typu urz�dze� musi by� bardzo niezawodne i
nie powinno wymaga� do pracy zbyt wiele pami�ci ini wydajnego
procesora, czyli element�w kt�re ze wzgl�du na rozmiary urz�dzenia
zwielokrotniaj� jego cen�. Oak powsta� na bazie do�wiadcze�
programist�w C++ i mia� podobne mo�liwo�ci, jednak by� o wiele
bardziej wra�liwy na b��dy programist�w. Z za�o�enia j�zyk Oak mia�
wyeliminowa� ca�kowicie b��dy w programach poprzez ograniczenie
mo�liwo�ci ich pope�niania. Poniewa� Oak nie toleruje b��d�w, s� one
wykrywane w procesie kompilacji, usuni�to r�wnie� niekt�re funkcje
jak np wska�niki i zarz�dzanie pami�ci�. Niestety w latach kiedy
powstawa� Oak rynek urz�dze� mobilnych praktycznie nie istnia�,
r�wnie� nie rozwin�� si� tak szybko jak przewidywali programi�ci,
dlatego te� nigdy nie sprzedano urz�dzenia korzystaj�cego z Oak.


 Z pomoc� tw�rcom Oak przyszed� Internet. gwa�townie rosn�ca
popularno�� internetu spowodowa�a zapotrzebowanie na oprogramowanie
do przegl�dania zasob�w globalnej sieci. W odpowiedzi firma Sun
zmieni�a nazw� Oak na Java i zaimplementowa�a "nowy" j�zyk w
przegl�darce o nazwie HotJava. Dodatkowo licencj� na j�zyk Java
wykupi�a firma Netscape kt�rej przegl�darka by�a w owym czasie
niekwestionowanym liderem na rynku.


Z biegiem czasu gwa�townie ros�y zasoby sprz�towe. Aby sprosta�
wymaganiom programist�w Windows tworz�cych bardzo zaawansowane
aplikacje, platforma Java stale si� rozszerza�a, pojawia�y si� nowe
funkcje, programowanie rozproszone, lepsze mechanizmy
bezpiecze�stwa.


Podczas gdy coraz wi�kszym zainteresowaniem cieszy�a si� przeno�no��
czyli inaczej mobilno�� urz�dze�, Java udost�pni�a now� rozszerzon�
platform� o nazwie Java 2. niezb�dne sta�o si� podzielenie platformy
Java 2 na kilka cz�ci. Postawowe funkcje uwa�anie za niezb�dne
minimum umieszczone zosta�y w jednym pakiecie i zyska�y nazw� Java
2 Standard Edition (J2SE). Do J2SE dodano kilka funkcji na potrzeby
okre�lonych zastosowa� korporacyjnych jak np. bezpieczna komunikacja
sieciowa, handel elektroniczny i tak powsta�a Java 2 Enterprise
Edition (J2EE).

Ze wzgl�du na urz�dzenia o ograniczonych mo�liwo�ciach, konieczne
sta�o si� ograniczenie tej rozbudowanej platformy Java.
Paradoksalnie Oak, prototyp j�zyka Java powsta� w�a�nie na potrzeby
urz�dze� mobilnych. Podczas gdy zapotrzebowanie na tego typu
platform� faktycznie zaistnia�o, platforma Java by�a ju� na tyle
rozbudowana �e nie mog�a zosta� zastosowana w urz�dzeniach
przeno�nych w takiej formie. Zatem programi�ci Java korzystaj�c z
do�wiadcze� z j�zykiem Oak stworzyli kilka platform o ograniczonych
mo�liwo�ciach. Powr�cono do platformy JDK 1.1 poprzednika Java 2 na
kt�rych oparto min. Java 2 Micro Edition (J2ME)


\section{Klasyfikacja CDC, CLDC i KVM}

Ze wzgl�du na r�nice w zasobach urz�dze� jak np. wydajno��
procesora, ilo�� pami�ci, spos�b nawi�zywania po��cze� J2ME
podzielona jest na dwie podstawowe konfiguracje CDC i CLDC.

\subsection{CLDC}
\emph{Connected Limited Device Configuration}

Konfiguracja ta jest przeznaczona dla najprostszych urz�dze�
elektroniki u�ytkowej. Typowa platforma CLDC to telefon kom�rkowy
lub notes wyposa�ony w oko�o 512kB pami�ci. Z tego powodu CLDC jest
�ci�le zwi�zany ze specyfikacj� bezprzewodowej Javy, kt�ra umo�liwia
instalowanie niewielkich aplikacji zwanych MIDletami. Wi�kszo�� firm
produkuj�cych telefony kom�rkowe podpisa�o porozumienie z firm� Java
Microsystems kt�re umo�liwia im stosowanie tej technologii, przez co
platforma Java mo�e by� z powodzeniem wykorzystywana przez prawie
wszystkie dost�pne na rynku modele telefon�w kom�rkowych i jej
popularno�� stale ro�nie.
\subsection{CDC}
\emph{Connected Device Configuration}

 Konfiguracja ta jest adresowana do urz�dze� znajduj�cych si�
 pomi�dzy urz�dzeniami CLDC a normalnymi systemami, na kt�rych mo�na
 uruchomi� J2SE (np. komputer klasy PC). Zwykle urz�dzenia takie
 maja wi�cej pami�ci (minimum 2MB) i nieco wydajniejszy procesor,
 dzi�ki czemu mog� obs�ugiwa� bardziej kompletne �rodowisko Java.
 Konfiguracja CDC jest stosowana w bardziej zaawansowanych notesach
 elektronicznych, smartfonach, telefonach sieciowych, urz�dzeniach
 sterowania oraz zestawach audio video.

\section{Profil informacji o urz�dzeniu przeno�nym, MIDP}

bla bla bla

\section{Struktura MIDletu}

\section{Media API}


    \subsection{Rejestrator}
    \label{sec:Rejestrator}
    \subsection{Kodowanie}
    \label{sec:Kodowanie}
    \subsection{Odtwarzanie}
    \label{sec:Odtwarzanie}

\section{Bluetooth API}
\label{sec:Bluetooth API}

    \subsection{Discovery Agent}
    \subsection{Nawi�zywanie po��czenia}
    \subsection{Streaming}
    \subsection{Przesy�anie danych}

\section{Uruchamianie}
\label{sec:Uruchamianie}

  \subsection{Kompilacja}
    \label{sec:Kompilacja}
    \subsection{Symulacja}
    \label{sec:Symulacja}

\section{Bezpiecze�stwo}


    \subsection{Autoryzacja}
    \label{sec:Autoryzacja}
    \subsection{Podpisywanie MIDlet�w}
    \label{sec:Podpisywanie MIDletow}

%
%\chapter{Aplikacja Walkie Talkie}
\label{sec:Aplikacja Walkie Talkie}

    \section{Jak to dzia�a}

  \section{Dane techniczne}

    \section{Mo�liwo�ci rozwoju}

%
%\chapter{Podsumowanie}
\label{sec:Podsumowanie}

%
%------------------------------------------------------------------------------
% <- Wykomentowaæ je¿eli praca nie ma dodatków
\appendix

%------------------------------------------------------------------------------
% Dodanie literatury
\newpage
\thispagestyle{empty}
\cleardoublepage
\addcontentsline{toc}{chapter}{Literatura}
\begin{thebibliography}{99}
%
\bibitem{har:mom}R.~F.~Harrington, \emph{Field Computation by
Moment Methods}, New York, USA: Macmillan, 1968.
%
\bibitem{rao}S. M. Rao, D. R. Wilton, and A. W. Glisson, ``Electromagnetic scattering by surfaces of arbitrary shape,'' \emph{IEEE Trans. Antennas Propag.}, vol. 30, no. 3, pp. 409--418, May 1982.
%
\bibitem{inman}M.~J.~Inman, and A.~Z.~Elsherbeni,~``Programming video cards for computational electromagnetics applications,'' \emph{Antennas Propag. Mag.}, vol. 47, pp. 71�78, December 2005.
%
\bibitem{killian}T.~Killian, D.~L.~Faircloth, and S.~M.~Rao, ``Acceleration of TM cylinder EFIE with CUDA,'' in  \emph{Antennas and Propagation Society International Symposium, APSURSI'09}, 2009, pp.1--4.
%
\bibitem{peng}S. Peng, and Z. Nie, ``Acceleration of the method of moments calculations by using graphics processing units,'' \emph{IEEE Trans. Antennas Propag.}, vol. 56, no. 7, pp. 2130--2133, July 2008.
%
\bibitem{lezar_1}E. Lezar, and D. Davidson, ``GPU acceleration of method of moments
matrix assembly using Rao-Wilton-Glisson basis functions,'' in \emph{Int. Conf.
on Electronics and Information Engineering, ICEIE2010}, 2010, Kyoto,
Japan, pp. V1-56--V1-60.
%
%\bibitem{hwu1}S. U. Hwu, D. R. Wilton, ``Electromagnetic scattering and radiation by
%arbitrary configurations of conducting bodies and wires,'' Dept. Elect. Eng.,
%Applied Electromagn. Lab., Univ. Houston, Tech. Rep. 87-17, 1987.
%
\bibitem{hwu2}S. U. Hwu, D. R. Wilton, S. M. Rao, ``Electromagnetic scattering and radiation by arbitrary conducting wire/surface configurations,'' \emph{Antennas and Propagation Society International Symposium, APS'88}, 1988, Syracuse, USA, pp.~890--893. 
%
\end{thebibliography}

%--------------------------------------------------------------------------------------------
\end{document}
